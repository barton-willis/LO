\documentclass[11pt]{article}
\usepackage[colorlinks=true,linkcolor=black,anchorcolor=black,citecolor=black,filecolor=black,menucolor=black,runcolor=black,urlcolor=black]{hyperref}
\usepackage[letterpaper, margin=0.75in]{geometry}
\usepackage[activate={true,nocompatibility},final,tracking=true,kerning=true,factor=1100,stretch=10,shrink=10]{microtype}
\frenchspacing
\usepackage[american]{babel}
\usepackage[T1]{fontenc}
\usepackage{utopia} %https://www.unk.edu/ccr/marketing-advertising/branding-and-identity-marks/typefaces.php
%\usepackage{lucidabr} 
%\usepackage{times}
%\usepackage{isomath}
%\usepackage{upgreek}
%\usepackage{enumerate}
\usepackage{amsmath}
\usepackage{enumitem}

\newenvironment{alphalist}{
\begin{enumerate}[label=(\arabic*),widest=107 ,leftmargin=25pt, itemsep=0pt]}
{\end{enumerate}}

\newenvironment{betalist}{
\begin{enumerate}[label=(\alph*),widest=zzz,leftmargin=25pt,itemsep=0pt]}
{\end{enumerate}}

\usepackage[super]{nth}
\newcommand\AM{a.m.} %https://www.unk.edu/ccr/news-internal-communication/writing-style-guide.php
\newcommand\PM{p.m.}

%\usepackage{multicol}
%\raggedright

\renewcommand{\thesubsection}{\arabic{subsection}}
\renewcommand{\thesubsubsection}{\arabic{subsection}.\arabic{subsubsection}}
%\renewcommand{\thesubsubsection}{}
%\renewcommand{\thechapter}{\arabic{chapter}.\arabic{subsection}}
\begin{document}
\begin{flushleft}
    \Large
\textbf{Learning Outcomes for Mathematics and Statistics Classes} \\
\vspace{0.25in}
\normalsize
\emph{Pending approval by the Department of Mathematics and Statistics}\\
\emph{Intended for the 2023-2024 Academic Year}
\end{flushleft}

\subsection{Introduction}

We begin with the Learning Outcomes for our LOPER 4 General Studies  
classes, followed by our Learning Outcomes for our two Experiential Learning
classes. After that, we list the Learning Outcomes for our developmental classes,
followed by our MATH and STAT classes in numerical order.


Finally, for our LOPER 4 General Studies classes, we give the class specific 
Learning Outcomes and detail the course concepts and assessments that are
used to satisfy the LOPER 4 General Studies Learning Outcomes.

%\begin{multicols}{2}
\subsection{General Studies Mathematics and Statistics Classes}

Our LOPER 4 General Studies classes are 
\textbf{MATH 102} (College Algebra),
\textbf{MATH 103} (Plane Trigonometry),
\textbf{MATH 106} (Mathematics for Liberal Arts),
\textbf{MATH 115} (Calculus I with Analytic Geometry),
\textbf{MATH 120} (Finite Mathematics),
\textbf{MATH 123} (Applied Calculus I),
\textbf{MATH 230} (Math for Elementary Teachers I),
\textbf{STAT 235} (Introduction to  Statistics for  Social Sciences),
and \textbf{STAT 241}  (Elementary Statistics). The learning outcomes for LOPER  4  classes are

\begin{alphalist}
\item Can describe problems using mathematical, statistical, or programming language.
\item Can solve problems using mathematical, statistical, or programming techniques.
\item Can construct logical arguments using mathematical, statistical, or programming concepts.
\item Can interpret and express numerical data or graphical information using 
   mathematical, statistical, or programming concepts and methods.
\end{alphalist}

\subsection{Experiential Learning Mathematics and Statistics Classes}


Our Experiential Learning (EL) classes are \textbf{MATH 300} 
(Tutoring in Mathematics) and \textbf{MATH 390} (Research Experience in Mathematics).
The learning outcomes for Experiential Learning classes are

\begin{alphalist}
    \item Student reflects critically on their experience, describing the experience including reactions, observations, and thoughts.

\item Student reflects critically on their experience, articulating connections between experiential learning activities and coursework.

\item Student communicates effectively verbally and/or in writing.

\item Student demonstrates dispositions appropriate to their chosen field.

\item Student demonstrates mastery of the practical use of skills within their chosen field.
\end{alphalist}
The course specific Learning Outcomes for MATH 300 are
\begin{alphalist}
\item Student reflects critically on their tutoring experience, describing the experience including reactions, observations, and thoughts on tutoring students in mathematics.

\item Student reflects critically on their tutoring experience, articulating connections between experiential learning in tutoring and their coursework in mathematics.

\item  Student communicates mathematical concepts, ideas, methods, and results effectively in oral form.

\item Student demonstrates dispositions appropriate to mathematics and tutoring of peers.

\item  Student demonstrates mastery of the practical use of mathematical skills and tutoring skills to support peers' learning.

\end{alphalist}
And for MATH 390, our course specific Learning Outcomes are
\begin{alphalist}
    \item  Student reflects critically on their research experience, describing the experience including reactions, observations, and thoughts on research problems in mathematics.
\item Student reflects critically on their research experience, articulating connections between experiential learning in mathematical research and their coursework in mathematics.
\item Student communicates mathematical concepts, ideas, methods, and results effectively in both oral and written form. 
\item Student demonstrates dispositions appropriate to mathematics and possibly other related areas (e.g., physics, computer sciences, etc.).
\item Student demonstrates mastery of the practical use of mathematical skills in solving research problems.
\end{alphalist}

\subsection{Learning Outcomes for Developmental Classes}

\subsubsection{MATH 90, Elementary Algebra}

On completion of this course, students will  be able to 
\begin{alphalist}
    \item Translate written sentences to algebraic expressions and solve applied problems.
    \item Understand arithmetic operations on whole numbers, fractions, decimals, and percents.
    \item Simplify algebraic expressions and solve linear and quadratic equations.
    \item Apply factoring techniques for quadratics and for the prime factorization of whole numbers.
    \item Identify, write, and graph linear equations and introduce the quadratic.
    \item Apply algebraic operations to polynomials and rational expressions.
\end{alphalist}

\subsubsection{MATH 101, Intermediate Algebra}

On completion of this course, students will  be able to 
\begin{alphalist}
\item  Solve linear equations and inequalities.
\item  Set up and solve applications with linear equations.
\item  Analyze functions including evaluating, finding domain, and graphing by translation.
\item  Apply operations to polynomial functions and rational functions.
\item  Factor polynomials of degree greater than two.
\item  Simplify rational functions and solve rational inequalities.
\item  Use the properties of exponents to simplify expressions.
\item  Solve radical equations.
\item  Analyze quadratic functions including finding the vertex and axis of symmetry.
\item Factor quadratic functions by completing the square and the quadratic formula.
\item  Graph quadratic functions.
\end{alphalist}
\subsubsection{MATH 104, Concepts in Mathematics and Statistics}


On completion of this course, students will be able to
\begin{alphalist}
    \item Identify, write, and graph linear functions.
    \item Write and graph quadratic functions.
    \item Solve linear and quadratic equations and inequalities.
    \item Solve two-variable systems of linear equations.
    \item Determine probabilities for independent events.
    \item Use the multiplication principle, permutations, and combinations.
    \item Use measures of center including mean, median, and mode.
    \item Use measures of variation including standard deviation, range, and variance.
    \item Describe distributions and create box plots.
\end{alphalist}

\subsection{Learning Outcomes for Mathematics Major Sequence Classes}

\subsubsection{MATH 202, Calculus II with Analytic Geometry}

 On completion of this course, students will be able to
 \begin{alphalist}
    \item use definite integrals to solve problems involving volume, arc length, 
       surface area, work, and center of mass. 
    \item use integration by parts, trigonometric substitution, and partial fractions 
        to evaluate definite and indefinite integrals.
    \item apply the concepts of limits, convergence, and divergence to evaluate 
        improper integrals.
    \item determine convergence or divergence of sequences and series.
    \item use Taylor and MacLaurin series to represent functions and integrate 
        functions.
    \item use parametrizations and polar coordinates to find areas and arc lengths.
 \end{alphalist}

\subsubsection{MATH 250, Foundations of Math}

On completion of this course, students will
\begin{alphalist}
    \item gain an understanding of na\"ive set theory. 
    \item gain an understanding of symbolic logic, quantifiers, and functions.
    \item gain an understanding of direct proofs, proofs by contradiction, proofs by contrapositive, and proofs by induction.
    \item gain the ability to read and understand mathematical proofs.
    \item gain the problem solving skills that are needed to create a mathematical proof.
\end{alphalist}

\subsubsection{MATH 251, Inquiry and Proof in 9--12 Mathematics}


On completion of this course, students will be able to
\begin{alphalist}
\item Articulate and utilize mathematical practices essential for 9--12 mathematics.  
\item Articulate the roles proof can play in secondary mathematics instruction.
\item Engage in mathematical inquiry using technological and mathematical tools.
\item Articulate mathematical arguments with precise mathematical language and symbols.
\item Communicate technical mathematical justifications in a manner appropriate for secondary students.
\item Determine how essential understanding of proof is embedded across mathematical content.
\end{alphalist}

\subsubsection{MATH 260, Calculus III}


On completion of this course, students will 
\begin{alphalist} 
    \item be able to use vectors to solve geometric problems and basic engineering statics problems.
    \item understand the concept of partial derivatives and limits of multivariable functions.
    \item understand the concept of a parameterized curve and understand the concept of curvature.
    \item be able to use partial derivatives to solve multivariable optimization problems.
    \item understand the concept of a line integral and apply it to basic physics problems involving energy and work.
    \item be able to set up and evaluate multiple integrals using Cartesian, polar, and spherical coordinates to find volumes, 
    surface areas, centroids, and moments of inertia.
    \item understand the concepts of the vector divergence, gradient, and curl in 
    three-dimensional space.
    \item understand the Green theorem and the Divergence theorem and be able to 
    apply these theorems to problems involving surface and line integrals.
\end{alphalist}

\subsubsection{MATH 270, Methods in Middle and High School Mathematics Teaching I}

On completion of this course, students will be able to
\begin{alphalist}
\item Thoroughly describe what is meant by ``doing, teaching, and learning'' mathematics in their own words and with the support of mathematics educational research.
\item Explain and provide real-life examples of the eight research-based mathematics teaching practices.
\item Identify and begin creating opportunities for high-quality instruction that includes mathematical discourse, productive struggle, purposeful questioning, and the connecting of multiple representations.
\item Articulate the essential mathematical concepts of 6--12 mathematics curriculum in regard to number, algebra/functions, statistics/probability, and geometry/measurement.
\item Explain the organization and benefits of the NCTM Standards, the Common Core State Standards, and the Nebraska State Standards.
\item Explain the history and current trends in mathematics education.
\item Define NCTM and NATM, explain membership benefits associated with each organization, and articulate the importance of professional affiliations.
\item Build upon foundational understanding of mathematics education and research-based mathematics teaching.
\item Be reflective of your own learning and realize how his/her own understanding influences student learning.
\end{alphalist}

\subsubsection{MATH 271, Field Experience in Middle and High School Mathematics I}

On completion of this course, students will be able to
\begin{alphalist}
\item Identify research-based mathematics teaching practices that are included in the classroom, as well as how they could be incorporated.
\item Engage 6--12 students in developmentally appropriate mathematical activities. 
\item Work with a diverse range of students individually, in small groups, and in large class settings.
\item Plan, facilitate, and reflect upon mathematical tasks that promote reasoning and sense making.
\item Collect and analyze data to determine if 6--12 students have built new knowledge.
\item Meet expectations of all Teacher Education Dispositions, including
\begin{betalist}
    \item Demonstrate effective oral communication skills
    \item Demonstrate effective written communication skills
    \item Demonstrate professionalism
    \item Demonstrate a positive and enthusiastic attitude
    \item Demonstrate preparedness in teaching and learning
    \item Exhibits an appreciation of and value for cultural and academic diversity
    \item Collaborates effectively with stakeholders
    \item Demonstrates self-regulated learner behaviors/takes initiative
    \item Exhibits the social and emotional intelligence to promote personal and educational goals/stability
\end{betalist}
\end{alphalist}

\subsubsection{MATH 280, Linear Algebra}
On completion of this course, students will
\begin{alphalist}
\item Be able to solve systems of linear equations using multiple methods, including row reduction to echelon form, row reduction to reduced 
echelon form, and multiplication by a matrix inverse.
\item Be able to carry out matrix operations, including finding sums, products, transposes, 
inverses,  and determinants of matrices.
\item Demonstrate an understanding of the concepts of vector space and subspace.
\item Demonstrate an understanding of linear independence, span, and basis.
\item Be able to determine eigenvalues and eigenvectors and solve eigenvalue problems.
\item Be able to apply the principles of matrix algebra to linear transformations.
\item Demonstrate an understanding of inner products and their associated norms.
\end{alphalist} 

\subsubsection{MATH 304, Introduction to Cryptography}

On successful completion of this course, students will be able to
\begin{alphalist}
\item use congruences, primes, the Chinese Remainder Theorem, Euler’s formula, quadratic reciprocity, group
theory, combinatorics, probability, and elliptic curves in cryptosystems and digital signature schemes,

\item understand the complexity of primality testing, factorization, discrete logarithm, and other important processes
in cryptography,

\item carry out relatively small examples of various encryption/decryption algorithms and digital signature schemes.

\end{alphalist}

\subsubsection{MATH 305, Differential Equations}

On completion of this class, students will
\begin{alphalist}
    \item Know the basic methods for solving first order differential equations (ODEs).
    \item Be able to set up and solve applied problems involving first order ODEs.
    \item Know the basic methods for solving second order ODEs.
    \item Know the basic methods for solving systems of first order ODEs.
    \item Be able to solve initial value problems for first and second order ODEs and for systems of ODEs.
\end{alphalist}

\subsubsection{MATH 310, College Geometry}

On completion of this course, students will be able to
\begin{alphalist}
    \item understand the basic definitions, axioms, and important theorems in neutral geometry.
    \item compare Euclidean geometry, hyperbolic geometry, and elliptical geometry.
    \item use the axiomatic or transformational approach to prove theorems in neutral/Euclidean geometry.
    \item use interactive geometry software for constructions in plane geometry, and
    explain geometric concepts and results in concrete models.
\end{alphalist}

\subsubsection{MATH 313, Graph Theory}

On completion of this course, students will be able to
\begin{alphalist}
    \item Understand the fundamental definitions and concepts of Graph theory.
    \item Understand trees and distance in a graph and including understanding Dijkstra's algorithm.
    \item Gain a basic understanding of matchings and graph factors
    \item Understand the main theorems about connectivity in graphs including 
    max  flow-min cut and the Ford Fulkerson algorithm.
    \item  Gain a basic understanding of graph colorings and how the 
    greedy coloring algorithm runs on interval graphs.
    \item Understand planarity and planarity testing algorithm from Demoucron, 
      Malgrange, and Pertuiset.
    \item  Gain a basic understanding about random graphs and probabilistic graph theory, including the Erd\'os-R\'enyi
       model.
\end{alphalist}

\subsubsection{MATH 330, Math for Elementary Teachers II}

On completion of this course, students will be able to
\begin{alphalist}
\item Explain and perform operations with fractions and decimals. 
\item Apply understanding of ratios, percentages, and proportions to real-life situations. 
\item Identify, categorize, compare and contrast various shapes and solids. 
\item Determine the area, surface area, and volume of two and three-dimensional objects. 
\item Approach mathematics problems using a variety of methods. 
\item Explain mathematical concepts to students at their level of understanding.
\end{alphalist}

\subsubsection{MATH 350,  Abstract Algebra}

On completion of this course
\begin{alphalist}
\item Students will be familiar with and able to grasp algebraic mathematical structures.
\item Students will be able to understand the concept of equivalence relation by applying different examples to the definition.
\item Students will be able to rigorously prove theorems in all the standard ways including using mathematical induction, some other direct method, and using proof by contradiction.
\item Students will have a firm grasp on Group theory including cyclic groups, permutation groups, homomorphisms, normal subgroups, and simple groups, with some real world applications of groups acting on a set. 
\item Students will have an understanding of the differences between rings, integral domains, unique factorization domains, Euclidean domains, and fields.
\item Students will comprehend correct proofs of formal statements and be able to formulate some of the proofs clearly and concisely.
\end{alphalist}

\subsubsection{MATH 365, Complex Analysis}

On completion of this course, students will 
\begin{alphalist}
    \item be able to represent complex numbers algebraically and geometrically.
    \item be able to use the definition of the limit to prove that a function has a limit.
    \item be able to find derivatives of complex valued functions from the limit definition.
    \item understand the connection between the complex exponential and the trigonometric functions and be able to 
        prove trigonometric identities using these connections.
    \item be able to use the Cauchy-Riemann equations to determine if a function is 
          analytic.
    \item be able to represent functions using Laurent and power series and be able 
           to find function residues, poles, and pole order.
    \item be able to evaluate contour integrals using residues.
    \item understand the Cauchy integral formula and its consequences,  including the 
       fundamental theorem of algebra.
\end{alphalist}





\subsubsection{MATH 399, Internship}
The Learning Outcomes are the following:
\begin{alphalist}
    \item Write a two page description of the activities of the 
    internship.
    \item Give three examples in which topics you learned in 
    undergraduate mathematics courses came up and/or were used in an 
    activity of the internship. Describe and discuss. 
    If three examples cannot be given, list one, or two, if possible, 
    and say that there were less than three.
    \item Give any examples of mathematical topics that came up 
    and/or were used in an activity of the internship that you had 
    not seen before. Describe and discuss. 
    \item Describe two jobs and/or careers for which the experience 
    of the internship made you more qualified than you were before.
    \item Fill in the blank: I would have been better prepared for 
    the internship if I had previously \underline{$\phantom{xxxxxx}$}.
    \item Fill in the blank: My next step(s) in developing a career path is(are) 
    \underline{$\phantom{xxxxxx}$}.
\end{alphalist}

\subsubsection{MATH 400, History of Mathematics}

On completion of this course, students will 
\begin{alphalist}
    \item understand the progression of mathematics through history.
    \item understand the history of a variety of mathematical topics.
    \item gain an understanding of several mathematicians through history.
\end{alphalist}

\subsubsection{MATH 404, Theory of Numbers}

On completion of this course, students will be able to
\begin{alphalist}
    \item explain the concepts of divisibility, prime number, and congruence. 
    \item calculate the greatest common divisor using the Euclidean algorithm and the prime factorization.
    \item solve linear congruence and quadratic congruence equations.
    \item understand Wilson's Theorem and Fermat's Little Theorem. 
    \item compute Euler's torsion function and other important multiplicative functions.
    \item use primitive roots and index arithmetic to solve higher-order congruence equations.  
    \item solve linear Diophantine equations and find primitive Pythagorean triples.
    \item understand how rational numbers are related to repeating decimals and continued fractions.
\end{alphalist}

\subsubsection{MATH 413, Discrete Mathematics}

On completion of MATH 413, students will 
\begin{alphalist}
    \item gain an understanding of counting principles and how to apply them.
    \item gain an understanding of discrete structures and how to use and analyze them, including induction, recursion, and probabilistic methods.
\end{alphalist}

\subsubsection{MATH 420, Numerical Analysis}

On completion of this course, students will
\begin{alphalist}
    \item understand IEEE arithmetic and know the rules for accurate computation.
    \item understand the concepts of linear and quadratic convergence and use these concepts to analyze 
        the efficiency of an algorithm.
    \item develop an understanding of the algorithms for solving linear and nonlinear equations, interpolation, 
       quadrature, least squares methods, and solution of differential equations.
    \item be able to use a programming language and graphical tools to solve problems numerically.
\end{alphalist}

\subsubsection{MATH 430,  Middle School Mathematics}

On completion of this course, students will be able to 
\begin{alphalist}
\item Conceptualize the real number system, including rational and irrational numbers. 
\item Explain algebraic procedures (i.e. solving equations/inequalities, laws of exponents). 
\item Simplify exponential and radical expressions. 
\item Apply transformation properties of congruent and similar figures. 
\item Identify and interpret graphs and functions. 
\item Compare and contrast different types of functions. 
\item Apply the Pythagorean Theorem to variety of situations. 
\item Use the coordinate plane to solve problems and display mathematics. 
\item Calculate, display, and interpret statistical measures. 
\item Perform probability simulations and interpret results. 
\item Approach mathematical problems using a variety of methods. 
\item Use various teaching models and techniques of curriculum delivery including effective questioning, cooperative learning, inquiry, and constructivist learning. 
\item Explain mathematical concepts to students at their level of understanding. 
\item Be reflective of your own learning and realize how his/her own understanding influences student learning.
\end{alphalist}

\subsubsection{MATH 445, Actuarial Science Seminar}

This course will help prepare the student to take the Actuarial exam P1.

\subsubsection{MATH 460, Advanced Calculus I}

On completion of this class, students will
\begin{alphalist}
    \item be able to prove basic propositions that involve the fundamentals of point set topology, including the concepts of open sets, closed sets, boundary points, and limit points.
     \item be able to prove basic propositions that involve the concept of the infimum and supremum. 
    \item demonstrate competence with basic properties of sequences including determining convergence and proving results involving the sum, difference, product, and quotient of sequences.
    \item be able to use the definitions of continuity, uniform continuity, the limit, and the derivative to prove basic propositions involving these concepts as well as be able to prove facts about specific functions.
    \item demonstrate the ability to use the Mean Value Theorem to prove theorems. 
    \item be able to define and evaluate the lower, upper, and general Riemann sums.
    \item demonstrate a solid understanding of the fundamental theorem of calculus.
\end{alphalist}

\subsubsection{MATH 465, Advanced Study in 9--12 Mathematics}

On  completion of this course, students will be able to
\begin{alphalist}

\item Connect higher level content knowledge to essential content in secondary mathematics.
\item Explain the impact of higher-level mathematical content knowledge on their teaching of high school students.
\item Articulate and utilize mathematical practices essential for 9--12 mathematics.
\item Describe essential understandings for 9--12 students in number theory, algebra/functions, statistics, probability, and calculus.
\item Demonstrate the interconnectedness of mathematics among mathematical ideas.
\item Utilize technological tools to explore essential mathematical content in number/quantity, algebra, statistics/probability, and calculus.
\end{alphalist}

\subsubsection{MATH 470, Methods in Middle and High School Mathematics Teaching II}

On completion of this course, students will be able to
\begin{alphalist}
\item Develop effective unit and lessons that support district, state and national standards and are developmentally appropriate.
\item Incorporate various forms of communication and connections (including within the subject area, to other disciplines and to real life) into lessons.
\item Diagnose and assess student performance in a variety of ways, including formative, summative, open ended, and performance assessments.
\item Address student diversity and various learning needs in lessons and units.
\item Use teaching methods and techniques of curriculum delivery including effective questioning, cooperative learning, inquiry, technology, and problem solving.
\item Incorporate various classroom organization and management techniques when teaching students.
\item Develop mathematical experiences for students, which will lead to positive dispositions toward math.
 \item Explain the organization and benefits of the NCTM Standards, the Common Core State Standards, Nebraska State Standards, and current trends in mathematics education.
\item Define NCTM and NATM, and explain membership benefits associated with each organization, and articulate the importance of professional affiliations.
\item Be reflective of your own learning and realize how his/her own understanding influences student learning.
\end{alphalist}

\subsubsection{MATH 471, Field Experience in Middle and High School Mathematics II}

On completion of this course, students will
\begin{alphalist}
\item Utilize research-based mathematics teaching practices in the classroom.
\item Engage 6--12 students in developmentally appropriate mathematics lessons.
\item Incorporate technology and tools into the 6--12 class in order to enhance mathematical understanding.
\item Work with a diverse range of students individually, in small groups, and in large class settings.
\item Plan, facilitate, and reflect upon mathematics lessons that promote reasoning and sense making.
\item Collect and analyze data to determine if 6--12 students have built new knowledge. 
\item Collaborate with colleagues, other school professionals, families, and stakeholders.
\item Continue to develop as a reflective practitioner.
\item Meet expectations of all Teacher Education Dispositions, including
\begin{betalist}
\item Demonstrate effective oral communication skills
\item Demonstrate effective written communication skills
\item Demonstrate professionalism
\item Demonstrate a positive and enthusiastic attitude
\item Demonstrate preparedness in teaching and learning
\item Exhibits an appreciation of and value for cultural and academic diversity
\item Collaborates effectively with stakeholders
\item Demonstrates self-regulated learner behaviors/takes initiative
\item Exhibits the social and emotional intelligence to promote personal and educational goals/stability
\end{betalist}
\end{alphalist}

\subsubsection{MATH 490, Special Topics in Mathematics}

The learning outcomes for MATH 490 vary by the course content.

\subsubsection{MATH 495, Independent Study in Mathematics}


The learning outcomes for MATH 495 vary by the course content.

\subsubsection{MATH 496, Mathematics Seminar}

The learning outcomes for MATH 496 vary by the course content.


\subsubsection{STAT 345, Applied Statistics I}

The class STAT 345 has not been offered in over five years.  The
department does not have a current syllabus for this course. Should we offer this course,  we will need to revise
the course description and create a syllabus. Accordingly, we have no Learning Outcomes for this class.


\subsubsection{STAT 399, Internship}

The Learning Outcomes are the following:
\begin{alphalist}
    \item Write a two page description of the activities of the internship.
    \item Give three examples in which topics you learned in 
    undergraduate mathematics courses came up and/or were used in an activity of the internship. Describe and discuss. If three examples cannot be given, list one, or two, if possible, and say that there were less than three.
    \item Give any examples of mathematical topics that came up and/or were used in an activity of the internship that you had not seen before. Describe and discuss. 
    \item Describe two jobs and/or careers for which the experience of the internship made you more qualified than you were before.
    \item Fill in the blank: I would have been better prepared for the internship if I had previously \underline{$\phantom{xxxxxx}$}.
    \item Fill in the blank: My next step(s) in developing a career path is(are) \underline{$\phantom{xxxxxx}$}
\end{alphalist}

\subsubsection{STAT 441, Probability and Statistics}

On completion of this course, students will be able to
\begin{alphalist}
	\item Demonstrate an understanding of the concepts of sample space, random variable, and probability of an event.
	\item Demonstrate an understanding of the axioms and basic theorems regarding probability measures.
	\item Demonstrate an understanding of conditional probability.
	\item Calculate the probability of an event in a discrete sample space when all the outcomes are equally likely using the sample-point method and basic counting techniques.
	\item Calculate the probability of an event in a sample space and when the outcomes are not equally likely using the event-composition method.
	\item Calculate the mean, variance, and standard deviation of an arbitrary discrete or continuous distribution.
	\item Demonstrate an understanding of the mathematical derivations of the formulas for the mean and variance of an assortment of special discrete and continuous distributions including the binomial, geometric, hypergeometric, Poisson, uniform, and normal distributions.
	\item Demonstrate an understanding of the similarities and differences between the cumulative distribution function of a discrete random variable and that of a continuous random variable.
	\item Demonstrate an understanding of moments and how moment generating functions for various distributions can be found and used to determine formulas for the mean and variance of distributions.
	\item Demonstrate an understanding of the mathematical derivations of an assortment of the techniques of statistical analysis, particularly in the areas of estimation (of means, variances, and standard deviations) and of hypothesis testing.
\end{alphalist}
\subsubsection{STAT 442, Mathematical Statistics}

The class STAT 442 has not been offered in over five years. The
department does not have a current syllabus for this course. Should we offer this course,  we will need to revise
the course description and create a syllabus. Accordingly, we have no Learning Outcomes for this class.

\subsubsection{STAT 495, Independent Study in Statistics}

The learning outcomes for STAT 495 vary by the course content.

%\end{multicols}


\subsection{\textbf{Addendum:} Math specific Learning Outcomes for LOPER 4 classes}

For our LOPER 4 General Studies classes, we give the course specific 
Learning Outcomes and we detail the course concepts and assessments that are
used to satisfy the LOPER 4 General Studies Learning Outcomes.

\subsubsection{MATH 102, College Algebra} 

In this course students will learn the concepts of relations and 
functions, properties and graphing of various types functions, and other topics such 
as systems of equations, matrices, and/or sequences and series. Students will 
investigate applied problems from various disciplines and 
algebraic principles used in the study of calculus, linear programming, and other 
areas of mathematics. Through this course, students will be able to

\begin{alphalist}

\item Solve algebraic equations and inequalities in one or two variables and illustrate these solutions graphically on the real number line and/or the rectangular coordinate system. 

\item Use the Pythagorean Theorem to develop distance computations and circular relationships in two variables on the coordinate system. 

\item   Understand the concepts of slope and the linear relationship, using graphic, algebraic, and verbal descriptions of this relationship. 

\item Apply the definition of the function, function notation, and the connections between graphic, algebraic, and verbal descriptions of functions. 

\item  Understand the domain and the range of a function, how to determine domain using equations and graphs of functions, and to determine range of a function by graph analysis. The students will be able to recognize other properties of several types of functions. 

\item   Understand the concepts of composite functions, one-to-one functions and inverse functions, and their importance in mathematics. 

\item  Sketch and analyze graphs of functions and use transformations.  The students will be able to analyze the relationships between the parent and the resulting functions using analytic and graphical techniques. These function types include general functions, linear functions, piecewise-defined functions, quadratic functions and quadratic inequalities, rational functions, polynomial functions, exponential functions, and logarithmic functions. 

\item   Develop an understanding of end behaviors of quadratic, rational, polynomial, exponential, and logarithmic functions. 

\item   Use functions algebraically, numerically, and graphical to model real life applications. 

\item   Set up and use systems of equations and inequalities to solve equations simultaneously.  

\item   Develop an understanding of sequences and series; be able to differentiate between geometric and arithmetic.  

\item   Appreciate other conic sections including the ellipse and hyperbola and their practical applications.  

\end{alphalist}

The LOPER 4 Learning Outcome `a' is broadly met mathematically 
in this course.  This includes developing the concepts of distance 
in the rectangular coordinate system (2), timing of projectiles (7, 8),  
population and natural phenomena modeling (3, 5, 6,7, 8, 9), and modeling 
economic and financial criteria using functions and mathematical reasoning (4, 5, 7).  
These are assessed via homework, quizzes, and exams where a portion of the grade 
for a solution is based on the mathematical set up, describing the problem, and 
defending the concluding results. 

The LOPER 4 Learning 
Outcomes `b' and `c' are achieved throughout the course through assigned homework, 
quizzes, exams, collaborative work and/or other assignments that are assessed by 
logically defending their solutions (1, 9, 10), and producing graphical, numerical, 
and/or algebraic support of their work (1, 2, 3, 4, 7, 8, 10, 11, 12).  

 The LOPER 4 Learning 
Outcome `d' is met directly (2, 3, 4, 7, 8, 10, 11, 12) and indirectly (5, 6, 9) 
in this course.  It is assessed by homework, quizzes, exams, and/or other work where 
a portion of the grade is based on the accuracy of their work and graphs and their 
interpretation of the data presented in the problems. 

\subsubsection{MATH 103, Plane Trigonometry} 
The first LOPER 4 Learning outcome (item `a') is met by 
the mathematical set up and preparation of solutions to various problems 
encountered in this course about trigonometric functions and 
trigonometric identities as well as their applications. The second 
and third learning outcomes (items `a' and `b') are met in solving 
such problems using 
mathematical skills and logical arguments. The fourth learning 
outcome (item `d') is met by understanding those problems which involve graphs 
and data and by giving solutions to those problems. 

The four learning 
outcomes are assessed by grading homework, quizzes, exams, and/or 
projects based on the set-up and defense of the submitted work, 
the validity of the submitted solution's logical reasoning, the 
accuracy of the answers, the accuracy of the graphs and data in the 
submitted solutions and/or the accuracy of the interpretation of the 
graphs and data from the assigned problem. 


\subsubsection{MATH 106, Mathematics for Liberal Arts} 
LOPER 4 Learning Outcome `a' will be achieved by describing how 
much power each person has in weighted voting systems by explaining 
quotas, veto power, dictators, dummies and other terms. 

Outcome `b' will be achieved by describing the best route to take 
using various algorithms such as brute force, cheapest link, and 
nearest neighbor algorithms. 

Outcome `c' will be achieved by using various fair division methods 
to divide assets in a fair and logical way so that everyone feels 
they received a fair share. 

Outcome `d' will be achieved by organizing the data into graphs such 
as bar graphs and histograms, and by finding the five number summary 
of the data and finding the standard deviation. 


\subsubsection{MATH 115, Calculus I with Analytic Geometry} 

Students will learn the concepts of continuity, 
the limit, the derivative, and the indefinite and definite integrals. 
Students will apply these concepts to problems involving the sciences 
and to applied problems of mathematics including geometry and the 
extreme values of functions. Specifically, the calculus specific learning outcomes 
(CSLO) are as follows: On completion of this class, students will

\begin{alphalist}
    \item  understand and compute limits and directional limits of functions in one variable and be able to determine (directional) continuity of such functions using the limit point definition of continuity.
    
    \item be able to discuss asymptotic behavior in terms of limits.

    \item understand the definition of derivative and its geometric interpretations.

    \item  compute derivatives using both the definition of derivative and derivative rules.

    \item use and apply the concept of the derivative as a rate of change to related rates, linearization, and differential problems.
    
    \item apply differentiation concepts through the mean value theorem and to curve sketching using concavity and intervals of increase/decrease, and to optimization and root finding (Newton's method).
    
    \item  understand the basics of anti-derivatives and integration.

    \item  develop a deeper understanding of summation notation and the relationship to integration and Riemann sums.
    
    \item compute basic definite integrals using Riemann sums and the fundamental theorem of calculus.

    \item compute basic integrals using integration by substitution techniques.

    \item  be able to set up and find the area between curves using integration.
    
    \end{alphalist}

Learning outcome `a' is met most directly by the mathematical set up 
and preparation of solutions to problems involving related rates and 
optimization in the calculus specific learning outcomes (CSLO) 5 
and 6. It is also met directly in discussion of asymptotic behavior 
CSLO 2. It is met indirectly through word problems related to many 
of the other CSLO. It is assessed by the grading of problems related 
to these CSLO on homework, quizzes, exams, and/or projects where a portion of the grade for a solution is based on the set-up and defend of the submitted work.
Learning outcomes `b' and `c' are closely related in this course and 
are met in solving problems related to every single CSLO. These are assessed via homework, quizzes, exams, and/or projects where a portion of the grade for a solution is based on the validity of the submitted solution’s logical reasoning and the accuracy of answers.
Learning outcome `d' is met most directly by CSLO 3, 6, and 11 as 
well as through reading and creating graphs and/or tabular data 
related all the CSLO. It is assessed by homework, quizzes, exams, 
and/or projects where a portion of the grade is based on the accuracy 
of the graphs and tables in the submitted solutions and/or the 
accuracy of the interpretation of such data from the assigned problem.

\subsubsection{MATH 120, Finite Mathematics} 

To meet LOPER 4 outcome `a,' students will define terms 
associated with matrices, probability, mathematics of finance, game 
theory, and logic. To meet outcome `b,' they will solve problems in 
systems of equations using the algebraic method and matrices; solve 
linear programming problems by simplex method, and geometric approach.
 To meet outcome `c,' they will discuss conditional statements and 
 equivalent statements, and form valid arguments. To meet outcome `d,'
  they will interpret the graph of the linear regression lines, and 
  the outcomes of game theory problems.  
 
\subsubsection{MATH 123, Applied Calculus I} 

LOPER 4 Outcome `a' will be satisfied through 
setting up and solving problems involving related rates, 
optimization, and curve sketching. Outcomes `b' and `c' are 
satisfied by nearly all the homework and exam problems and 
students will be evaluated through grading of the homework and 
exams. Outcome `d' is achieved through curve sketching, geometrical 
definitions of limits and derivatives, and finding area using 
integration. 

 \subsubsection{MATH 230, Math for Elementary Teachers I} 
 
 On completion of this course, students will be able to
 \begin{alphalist}

    \item Explain and apply the problem-solving process, using a variety of problem-solving strategies. 

    \item Perform mathematical operations in varied bases. 

    \item Explain and use traditional and non-traditional algorithms. 

    \item Identify specific qualities of a number and number relations. 

    \item Demonstrate integer operations using a variety of strategies. 

    \item Approach mathematics problems using a variety of methods. 

    \item Explain mathematical concepts to students at their level of understanding. 

    \item Represent numerical data in graphical forms and with descriptive statistics. 
 \end{alphalist}

 LOPER 4 Outcome `a' will be achieved as students 
 engage in a variety of instructional activities aimed at meeting 
 the Course Specific Objectives (CSO) 1, 3, 5, 6, and 7. For example, 
 students work in groups in class to translate word problems into 
 pictorial and symbolic representations in order to meet CSO 1. 
 They also practice using correct mathematical terminology when 
 explaining the foundations of a variety of computational algorithms 
 in pairs in class to meet CSO 3. Competence on this GS outcome is 
 assessed throughout the course using some problems on Practice and 
 Preparation assignments, all Teaching Tasks, and all Problem-Solving 
 Tasks.  

 

 Outcome `b' will be achieved as students engage in a variety of 
 instructional activities aimed at meeting the CSO 1-6. In class, 
 students are given opportunities to approach problems using 
 mathematical techniques every class session. Their understanding 
 of these techniques is supported by appropriate instruction 
 regarding the appropriate contexts for the techniques, the 
 foundation ensuring the efficacy of the techniques, and ways to 
 help their future students learn to use the same techniques. 
 Competence on this GS outcome is assessed throughout the course 
 using most problems on Practice and Preparation assignments, 
 most problems on Cumulative Tasks, and all Problem-Solving Tasks. 
 
  
 
 Outcome `c' will be achieved as students engage in a variety of 
 instructional activities aimed at meeting the CSO 3, 4, 6, and 7. 
 For example, when we use computational algorithms to solve problems, 
 students do not merely describe the procedural steps in performing a 
 computational algorithm, but are required to provide mathematical 
 reasoning that supports the efficacy and justifies the accuracy of 
 the algorithm. We support their explanations by allowing students 
 to wrestle with their own ideas and providing well-timed direct 
 instruction to help them build on their explanations to achieve 
 more mathematically sound justifications. Competence on this GS 
 outcome is assessed throughout the course using some problems on 
 Practice and Preparation assignments, some problems on Cumulative 
 Tasks, all Teaching Tasks, and all Problem--Solving Tasks. 
 
   
 Outcome `d' will be achieved as students engage in instructional 
 activities aimed at meeting the CSO 8. Students will learn how to 
 calculate descriptive statistics and represent data in a variety of 
 graphical forms in class or through online learning modules. 
 We help students to interpret descriptive statistics in context 
 and identify conceptual understanding of the computed measures. 
 Competence on this GS outcome is assessed on the third teaching 
 task and/or Practice and Preparation assignments.  

 \subsubsection{STAT 235, Introduction to Statistics for Social Sciences} 
 
 The class specific Learning Outcomes are

 \begin{alphalist}
 \item Learn to construct frequency distributions. Learn to graph 
 and interpret quantitative data  sets using histograms, frequency 
 polygons, relative frequency histograms, and ogives.

\item Learn to graph and interpret qualitative data sets using 
pie charts.

\item Learn how to find the mean, median and mode of a data set.

\item Learn how to find a weighted mean and estimate the mean of 
grouped data. 

\item Be able to describe the shapes of a distribution as symmetric, 
uniform, or skewed and  how to compare the mean and median for each. 

\item Learn how to find the range, variance, standard deviation, 
quartiles, percentiles and  standard score of a data set and use 
the information interpret the data and make logical arguments 
concerning the data.

\item How to identify the sample space of a probability experiment 
and identify simple  events.

\item How to find the probability of one or two events, the 
probability of two or more events occurring in sequence and how to 
find conditional probabilities. 

\item How to construct and graph a discrete probability distribution 
and how to compute the   mean, variance, standard deviation and 
expected value of a discrete probability  distribution.

\item How to find binomial probabilities using the binomial 
probability formula.

\item How to construct, graph, and find the mean, variance, and 
standard deviation of a  binomial probability distribution.

\item Learn how to interpret graphs and find the areas under the 
standard normal curve.

\item Learn how to find probabilities for normally distributed 
variable using a table and/or  technology.

\item Learn how to find a z--score and transform a z--score to a 
raw data value.

\item Learn to apply the Central Limit Theorem to find the 
probability of a sample mean.

\item How to use a normal distribution to approximate binomial 
probabilities.

\item Learn how to construct and interpret confidence intervals 
for a mean, proportion,  variance and standard deviation.

\item How to determine the minimum sample size required when 
estimating a population mean  or proportion.

\item Use inferential statistics to analyze situation by preforming 
hypothesis tests on one or  two samples. 

\item Use hypothesis tests in order to test claims about population 
parameters including means,  proportion, variances, and standard 
deviations.

\item Use the Goodness--of--fit test and the chi--square 
distribution to test whether a frequency distribution fits an 
expected distribution.
 \end{alphalist}

 The first two LOPER 4 Learning Outcomes (items `a' and `b') of 
 describing and solving problems using statistical/mathematical 
 language and techniques are used throughout the course and most 
 specifically through course specific outcomes 3, 4, 6, 7, 8, 9, 
 10, 13, 14, 15, 16, and 18 listed below.  These learning outcomes 
 involve the set-up and solutions to such concepts as the measures 
 of central tendency, measures of variation, measures of position, 
 basic concepts of probability, and the binomial distributions.  

 

 The third LOPER 4 Learning Outcome (item `c') is most directly met 
 through inferential statistics and the process of constructing and 
 interpreting confidence intervals and through hypothesis testing. 
 Course specific outcomes that met this learning outcome would 
 include 6, 17, 19, 20, and 21. 
 
 The final LOPER 4 Learning Outcome (item `d') of interpreting and 
 expressing numerical data or graphical information is directly met 
 through methods used in descriptive statistics and the processes 
 of  constructing and interpreting frequency distributions, 
 frequency histograms, frequency polygons, relative frequency 
 histograms, ogives and pie charts.  Course specific outcomes 
 that met this learning outcome include 1, 2, 5, 11, and 12. 
  
 
 All four learning outcomes are assessed by grading homework, 
 quizzes, exams and extra course work as may be assigned to 
 better the understanding of individual concepts that may pose 
 difficulty throughout the term.  All assessment is based on the 
 accuracy of the work and the interpretation of the results. 

 \subsubsection{STAT 241,  Elementary Statistics }  
 
 LOPER 4 Learning Outcomes `a,' `b,' `c,' and 
 `d' will be achieved by requiring the students to work problems of 
 a statistical nature on homework and exams.  In particular, 
 these problems will require students to

 \begin{alphalist}
    

    \item understand and use accepted statistical language and 
    symbols (outcome `a').

    \item solve problems using standard statistical techniques 
    (outcome `b'). 

    \item clearly communicate their work/procedure and/or explain 
    the argument which led them to their answer (outcome `c'). 
    (This is especially true of exam and select handwritten 
    homework problems.)

    \item interpret the numerical result(s) of a computation or 
    procedure in the context of the real--world scenario in which 
    the problem originated (This interpretation will often 
    require the students to say what their numerical answer 
    means in a short, clear English sentence.) (outcome `d').   

 \end{alphalist}
LOPER 4 Outcome `d' will also be achieved by requiring students to
\begin{alphalist}

    \item calculate the mean, median, mode, variance, and standard 
    deviation of numerical data sets. 

    \item create histograms, frequency polygons, ogives, and pie 
    charts to depict/summarize data sets. 

    \item use tables and graphs describing various binomial, 
    hypergeometric, geometric, and Poisson distributions to 
    solve problems.

    \item use tables and graphs describing various normal 
    distributions, t--distributions, and chi-squared distributions 
    to solve problems. 
\end{alphalist}
 
\end{document}


\noindent \textbf{Unit 1}
\begin{alphalist}
    \item Evaluate algebraic expressions
    \item Translate English phrases/sentences into algebraic expressions
    \item Determine whether a number is a solution of an equation
    \item Evaluate formulas
    \item Convert between mixed numbers and improper fractions
    \item Write the prime factorization of a composite number
    \item Reduce or simplify fractions
    \item Add, subtract, multiply and divide fractions
    \item Solve problems involving fractions in algebra
    \item Define the sets that make up the real numbers
    \item Graph numbers on a number line
    \item Express rational numbers as decimals
    \item Classify numbers as belonging to one or more sets of the real numbers
    \item Understand and use inequality symbols
    \item Find the absolute value of a real number
    \item Understand and use the vocabulary of algebraic expressions
    \item Use the commutative, associate, and distributive properties
    \item Simplify algebraic expressions
    \item Understand and use the vocabulary of algebraic expressions
    \item Add numbers with or without a number line
    \item Add, subtract, multiply and divide real numbers
    \item Use identity and inverse properties for addition and multiplication 
    \item Use the basic operations to simplify algebraic expressions 
    \item Solve applied problems using a series of basic operations
    \item Evaluate/simplify exponential expressions
    \item Use the order of operations agreement
    \item Evaluate mathematical models
\end{alphalist}
\noindent \textbf{Unit 2}
 
\begin{alphalist}
    \item Identify linear equations in one variable
    \item Use the addition and multiplication properties of equality to solve equations
    \item Solve linear equations
    \item Solve linear equations involving fractions and decimals
    \item Identify equations with no solution or infinitely many solutions
    \item Solve applied problems using formulas
    \item Solve a formula for a variable
    \item Use the percent formula
    \item Solve applied problems involving percent change
    \item Translate English phrases into algebraic expressions
    \item Solve algebraic word problems using linear equations
    \item Solve problems using formulas for perimeter, the circumference of a circle, area, and volume
    \item Solve problems involving the angles of a triangle
    \item Solve problems involving complementary and supplementary angles
\end{alphalist}
\noindent \textbf{Unit  3}
 \begin{alphalist}
    \item Plot/find coordinates of points in the rectangular coordinate system
    \item Determine whether an ordered pair is a solution of an equation
    \item Find solutions of an equation in two variables
    \item Use point plotting to graph linear equations
    \item Use graphs of linear equations to solve problems
    \item Use a graph to identify intercepts
    \item Graph a linear equation in two variables using intercepts
    \item Graph horizontal or vertical lines
    \item Compute the slope of a line
    \item Use slope to show that lines are parallel or perpendicular
    \item Calculate rate of change in applied situations
    \item Find the slope of a line and its y-intercept from its equation
    \item Graph lines in slope-intercept form
    \item Use slope and the y-intercept to graph $Ax + By = C$
    \item Use the slope and y-intercept to model data
    \item Use the point-slope form to write equations of a line
    \item Write linear equations that model data and make predictions
 \end{alphalist}
\noindent \textbf{Unit 4}
 \begin{alphalist}
    \item Understand the vocabulary used to describe polynomials
    \item Add and subtract polynomials
    \item Graph equations defined by polynomials of degree two
    \item Use FOIL in polynomial multiplication
    \item Find the square of a binomial sum or difference
    \item Multiply polynomials
    \item Add, subtract, and multiply polynomials in several variables
    \item Evaluate polynomials in several variables
    \item Divide monomials
    \item Check polynomial division
    \item Divide a polynomial by a monomial 
    \item Divide polynomials by binomials
    \item Use the negative exponent rule
    \item Simplify exponential expressions
 \end{alphalist}
\noindent \textbf{Unit 5}
 \begin{alphalist}
    \item Find the greatest common factor and factor it out of a polynomial
    \item Factor by grouping
    \item Factor trinomials of the form $x^2 + bx + c$
    \item Factor trinomials by trial and error
    \item Factor trinomials by grouping
    \item Factor the difference of two squares
    \item Factor perfect square trinomials
    \item Factor the sum or difference of two cubes
    \item Use a general strategy to recognize the appropriate method of factoring a polynomial
    \item Solve quadratic equations by factoring
    \item Solve problems using quadratic equations
 \end{alphalist}
\noindent \textbf{Unit 6}
 \begin{alphalist}
    \item Find numbers for which a rational expression is undefined
    \item Simplify rational expressions
    \item Solve applied problems involving rational expressions
    \item Multiply and divide rational expressions
    \item Add and subtract rational expressions
    \item Simplify complex rational expressions by dividing
    \item Simplify complex rational expressions by multiplying by the LCD
    \item Solve rational equations
    \item Solve problems involving formulas with rational expressions
 \end{alphalist}
\noindent \textbf{Unit 7}
 \begin{alphalist}
    \item Solve quadratic equations using the square root property
    \item Solve problems using the Pythagorean Theorem
    \item Find the distance between two points
    \item Complete the square of a binomial
    \item Solve quadratic equations by completing the square
    \item Solve quadratic equations using the quadratic formula
    \item Solve problems using quadratic equations
 \end{alphalist}

 The Learning Outcomes for MATH 101 are

\noindent \textbf{Unit 1}
\begin{alphalist}
    \item Determine whether a number is a solution to a linear equation
    \item Solve linear equations using the properties of equality
    \item Identify identities and contradictions
    \item Use geometric formulas to find the perimeter of plane two-dimensional figures
    \item Find the circumference of a circle
    \item Use formulas to find the area and volume of various geometric figures
    \item Solve equations for a specific variable
    \item Solve application problems using formulas
    \item Use problem solving techniques to solve percent, investment, uniform motion, and mixture problems.
\end{alphalist}
\noindent \textbf{Unit 2}
\begin{alphalist}
    \item Plot ordered pairs and determine the coordinates of a point
    \item Read graphs and graph paired data
    \item Find the midpoint of a line segment
    \item Determine whether an ordered pair is a solution of an equation
    \item Graph linear equations in two variables
    \item Use linear models to solve applied problems
    \item Calculate an average rate of change
    \item Find the slope of a line using a graph or the slope formula
    \item Use slope to solve application problems
    \item Determine whether lines are parallel or perpendicular using slope
    \item Use the slope-intercept and point-slope forms to write equations of lines
    \item Identify functions and use function notation
    \item Use the vertical line test to identify functions
    \item Find the domain and range of functions
    \item Graph linear functions
    \item Find the domain and range of functions graphically
    \item Graph nonlinear functions
    \item Translate and reflect graphs of functions
\end{alphalist}
 
\noindent \textbf{Unit 3}
\begin{alphalist}
    \item Read and interpret inequality symbols
    \item Graph intervals and use interval and set-builder notation
    \item Solve linear inequalities using properties of inequality
    \item Use linear inequalities to solve problems
    \item Find the intersection and union of two sets
    \item Solve double linear inequalities
    \item Solve compound inequalities containing the words and or
    \item Graph linear inequalities in two variables
    \item Solve applied problems involving linear inequalities in two variables
    \item Solve systems of linear inequalities
    \item Graph compound inequalities
    \item Solve problems involving systems of linear inequalities
\end{alphalist}
\noindent \textbf{Unit 4}
\begin{alphalist}
    \item Identify bases and exponents
    \item Use the exponent rules to simplify expressions
    \item Define and classify polynomials
    \item Evaluate polynomial functions
    \item Find function values and the domain and range of polynomial functions graphically
    \item Add, subtract, and multiply polynomials
    \item Find special products
    \item Use multiplication to simplify expressions
    \item Find the greatest common factor of a list of terms
    \item Factor out the greatest common factor
    \item Factor by grouping
    \item Use factoring to solve formulas for a specific variable
    \item Factor perfect square trinomials
    \item Factor trinomials in the form $x^2 + bx + c$ and $ax^2 + bx + c$
    \item Use substitution to factor trinomials
    \item Use the grouping method to factor trinomials
    \item Factor the difference of two squares
    \item Factor the sum and difference of two cubes
    \item Solve higher-degree polynomial equations by factoring
    \item Use quadratic equations to solve problems
\end{alphalist}
 
\noindent \textbf{Unit 5}
\begin{alphalist}
    \item Define rational expressions and functions
    \item Evaluate rational functions
    \item Find the domain of a rational function
    \item Recognize the graphs of rational functions
    \item Simplify rational expressions
    \item Multiply and divide rational expressions
    \item Perform mixed operations on rational expressions
    \item Find the least common denominator of rational expressions
    \item Add and subtract rational expressions
    \item Simplify complex fractions using division
    \item Simplify complex fractions using the LCD
    \item Divide a polynomial by a monomial
    \item Divide a polynomial by a polynomial
    \item Divide polynomials with missing terms 
    \item Solve rational equations
    \item Solve rational equations with extraneous solutions
    \item Solve formulas for a specific variable
    \item Solve shared-work and uniform-motion problems
\end{alphalist}

\noindent \textbf{Unit 6}
\begin{alphalist}
    \item Find square, cube and $n^\mathrm{th}$  roots
    \item Graph the square root and cube root functions
    \item Evaluate radical functions
    \item Convert between radicals and rational exponents
    \item Use rules of exponents to simplify expressions
    \item Simplify radical expressions by using prime factorization, and the product and quotient rule
    \item Add and subtract radical expressions
    \item Rationalize numerators and denominators of radical expressions
    \item Multiply and divide radical expressions
    \item Solve equations containing one and two radicals
    \item Solve formulas containing radicals
    \item Use the Pythagorean Theorem to solve problems 
\end{alphalist}
\noindent \textbf{Unit 7}
\begin{alphalist}
    \item Use the square root property to solve quadratic equations
    \item Solve quadratic equations by completing the square
    \item Derive the quadratic formula
    \item Solve quadratic equations using the quadratic formula
    \item Use the quadratic formula to solve application problems
    \item Use the discriminant to determine the number and type of solutions to quadratic equations
    \item Solve application problems using quadratic equations
    \item Find the vertex of a quadratic function using $-b/2a$
    \item Graph quadratic functions
    \item Graph functions of the form $f(x) = ax^2 + bx + c$ by completing the square
    \item Determine the minimum and maximum values of quadratic functions
    \item Solve quadratic equations graphically
    \item Solve quadratic inequalities
    \item Solve rational inequalities
    \item Graph nonlinear inequalities in two variables
\end{alphalist}

\noindent The online MATH  101 course is slightly different. The online course does additional work in Unit 1, and
it eliminates Unit 3. The Learning Outcomes for the additional work in Unit 1 are
 
\begin{alphalist}
    \item Write verbal and mathematical models
    \item Use equations to construct tables of data
    \item Define the set of natural numbers, whole numbers, integers, rational numbers, irrational numbers, and real numbers.
    \item Graph real numbers
    \item Order the real numbers
    \item Find the additive inverse and the absolute value of real numbers
    \item Add, subtract, multiply, and divide real numbers
    \item Find powers and square roots of real numbers
    \item Use the order of operations rule
    \item Evaluate algebraic expressions
    \item Identify terms, factors, and coefficients
    \item Identify and use properties of real numbers
    \item Simplify algebraic expressions using the properties of real numbers

\end{alphalist}


\end{document}