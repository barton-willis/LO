\documentclass[11pt]{article}
\usepackage[colorlinks=true,linkcolor=black,anchorcolor=black,citecolor=black,filecolor=black,menucolor=black,runcolor=black,urlcolor=black]{hyperref}
\usepackage[letterpaper, margin=0.75in]{geometry}
\usepackage[activate={true,nocompatibility},final,tracking=true,kerning=true,factor=1100,stretch=10,shrink=10]{microtype}
\frenchspacing
\usepackage[american]{babel}
\usepackage[T1]{fontenc}
\usepackage{utopia} %https://www.unk.edu/ccr/marketing-advertising/branding-and-identity-marks/typefaces.php
\usepackage{isomath}
\usepackage{upgreek}
%\usepackage{enumerate}
\usepackage{amsmath}
\usepackage{enumitem}

\newenvironment{alphalist}{
\begin{enumerate}[label=(\arabic*),widest=107 ,leftmargin=10pt, itemsep=0pt]}
{\end{enumerate}}

\newenvironment{betalist}{
\begin{enumerate}[label=(\alph*),widest=zzz,leftmargin=20pt,itemsep=0pt]}
{\end{enumerate}}

%\newenvironment{betalist}{
%\begin{enumerate}[label=(\arabic*),widest=27,itemindent=*,leftmargin=0pt]}
%{\end{enumerate}}

%\newenvironment{alphalist}{
 % \begin{enumerate}[(1)]
 %   \addtolength{\itemsep}{-1.25\itemsep}}
 % {\end{enumerate}}

 % \newenvironment{betalist}{
 % \begin{enumerate}[(a)]
 %   \addtolength{\itemsep}{-1.25\itemsep}}
%  {\end{enumerate}}

\newcommand\AM{a.m.} %https://www.unk.edu/ccr/news-internal-communication/writing-style-guide.php
\newcommand\PM{p.m.}

%\usepackage{blindtext}
%\usepackage{multicol}

\begin{document}
\begin{flushleft}
\textbf{Learning Outcomes for Mathematics and Statistics Classes} \\
\vspace{0.1in}
\emph{Approved by the Department of Mathematics and Statistics on XXX}
\end{flushleft}

%\begin{multicols}{2}
\subsubsection*{Learning Outcomes for General Studies Mathematics and Statistics Classes (Loper 4)}

Our General Studies classes are:
\textbf{MATH 102} (College Algebra),
\textbf{MATH 103} (Plane Trigonometry),
\textbf{MATH 106} (Mathematics for Liberal Arts),
\textbf{MATH 115} (Calculus I with Analytic Geometry),
\textbf{MATH 120} (Finite Mathematics),
\textbf{MATH 123} (Applied Calculus I),
\textbf{MATH 230} (Math for Elementary Teachers I),
\textbf{STAT 235} (Introduction to  Statistics for  Social Sciences),
and \textbf{STAT 241}  (Elementary Statistics). The learning outcomes for Loper 4  classes are:

\begin{alphalist}
\item Can describe problems using mathematical, statistical, or programming language.
\item Can solve problems using mathematical, statistical, or programming techniques.
\item Can construct logical arguments using mathematical, statistical, or programming concepts.
\item Can interpret and express numerical data or graphical information using 
   mathematical, statistical, or programming concepts and methods.
\end{alphalist}

\subsubsection*{MATH 90, Elementary Algebra}

Learning Outcomes:

\noindent Chapter 1:
\begin{alphalist}
    \item Evaluate algebraic expressions
    \item Translate English phrases/sentences into algebraic expressions
    \item Determine whether a number is a solution of an equation
    \item Evaluate formulas
    \item Convert between mixed numbers and improper fractions
    \item Write the prime factorization of a composite number
    \item Reduce or simplify fractions
    \item Add, subtract, multiply and divide fractions
    \item Solve problems involving fractions in algebra
    \item Define the sets that make up the real numbers
    \item Graph numbers on a number line
    \item Express rational numbers as decimals
    \item Classify numbers as belonging to one or more sets of the real numbers
    \item Understand and use inequality symbols
    \item Find the absolute value of a real number
    \item Understand and use the vocabulary of algebraic expressions
    \item Use the commutative, associate, and distributive properties
    \item Simplify algebraic expressions
    \item Understand and use the vocabulary of algebraic expressions
    \item Add numbers with or without a number line
    \item Add, subtract, multiply and divide real numbers
    \item Use identity and inverse properties for addition and multiplication 
    \item Use the basic operations to simplify algebraic expressions 
    \item Solve applied problems using a series of basic operations
    \item Evaluate/simplify exponential expressions
    \item Use the order of operations agreement
    \item Evaluate mathematical models
\end{alphalist}
\noindent Chapter 2:
 
\begin{alphalist}
    \item Identify linear equations in one variable
    \item Use the addition and multiplication properties of equality to solve equations
    \item Solve linear equations
    \item Solve linear equations involving fractions and decimals
    \item Identify equations with no solution or infinitely many solutions
    \item Solve applied problems using formulas
    \item Solve a formula for a variable
    \item Use the percent formula
    \item Solve applied problems involving percent change
    \item Translate English phrases into algebraic expressions
    \item Solve algebraic word problems using linear equations
    \item Solve problems using formulas for perimeter, the circumference of a circle, area, and volume
    \item Solve problems involving the angles of a triangle
    \item Solve problems involving complementary and supplementary angles
\end{alphalist}
\noindent Chapter 3:
 \begin{alphalist}
    \item Plot/find coordinates of points in the rectangular coordinate system
    \item Determine whether an ordered pair is a solution of an equation
    \item Find solutions of an equation in two variables
    \item Use point plotting to graph linear equations
    \item Use graphs of linear equations to solve problems
    \item Use a graph to identify intercepts
    \item Graph a linear equation in two variables using intercepts
    \item Graph horizontal or vertical lines
    \item Compute the slope of a line
    \item Use slope to show that lines are parallel or perpendicular
    \item Calculate rate of change in applied situations
    \item Find the slope of a line and its y-intercept from its equation
    \item Graph lines in slope-intercept form
    \item Use slope and the y-intercept to graph $Ax + By = C$
    \item Use the slope and y-intercept to model data
    \item Use the point-slope form to write equations of a line
    \item Write linear equations that model data and make predictions
 \end{alphalist}
\noindent Chapter 5:
 \begin{alphalist}
    \item Understand the vocabulary used to describe polynomials
    \item Add and subtract polynomials
    \item Graph equations defined by polynomials of degree two
    \item Use FOIL in polynomial multiplication
    \item Find the square of a binomial sum or difference
    \item Multiply polynomials
    \item Add, subtract, and multiply polynomials in several variables
    \item Evaluate polynomials in several variables
    \item Divide monomials
    \item Check polynomial division
    \item Divide a polynomial by a monomial 
    \item Divide polynomials by binomials
    \item Use the negative exponent rule
    \item Simplify exponential expressions
 \end{alphalist}
\noindent Chapter 6:
 \begin{alphalist}
    \item Find the greatest common factor and factor it out of a polynomial
    \item Factor by grouping
    \item Factor trinomials of the form $x^2 + bx + c$
    \item Factor trinomials by trial and error
    \item Factor trinomials by grouping
    \item Factor the difference of two squares
    \item Factor perfect square trinomials
    \item Factor the sum or difference of two cubes
    \item Use a general strategy to recognize the appropriate method of factoring a polynomial
    \item Solve quadratic equations by factoring
    \item Solve problems using quadratic equations
 \end{alphalist}
\noindent Chapter 7:
 \begin{alphalist}
    \item Find numbers for which a rational expression is undefined
    \item Simplify rational expressions
    \item Solve applied problems involving rational expressions
    \item Multiply and divide rational expressions
    \item Add and subtract rational expressions
    \item Simplify complex rational expressions by dividing
    \item Simplify complex rational expressions by multiplying by the LCD
    \item Solve rational equations
    \item Solve problems involving formulas with rational expressions
 \end{alphalist}
\noindent Chapter 9:
 \begin{alphalist}
    \item Solve quadratic equations using the square root property
    \item Solve problems using the Pythagorean Theorem
    \item Find the distance between two points
    \item Complete the square of a binomial
    \item Solve quadratic equations by completing the square
    \item Solve quadratic equations using the quadratic formula
    \item Solve problems using quadratic equations
 \end{alphalist}

\subsubsection*{Learning Outcomes for MATH 101 (Intermediate Algebra)}


\noindent Chapter 1:
\begin{alphalist}
    \item Determine whether a number is a solution to a linear equation
    \item Solve linear equations using the properties of equality
    \item Identify identities and contradictions
    \item Use geometric formulas to find the perimeter of plane two-dimensional figures
    \item Find the circumference of a circle
    \item Use formulas to find the area and volume of various geometric figures
    \item Solve equations for a specific variable
    \item Solve application problems using formulas
    \item Use problem solving techniques to solve percent, investment, uniform motion, and mixture problems.
\end{alphalist}
\noindent Chapter 2:
\begin{alphalist}
    \item Plot ordered pairs and determine the coordinates of a point
    \item Read graphs and graph paired data
    \item Find the midpoint of a line segment
    \item Determine whether an ordered pair is a solution of an equation
    \item Graph linear equations in two variables
    \item Use linear models to solve applied problems
    \item Calculate an average rate of change
    \item Find the slope of a line using a graph or the slope formula
    \item Use slope to solve application problems
    \item Determine whether lines are parallel or perpendicular using slope
    \item Use the slope-intercept and point-slope forms to write equations of lines
    \item Identify functions and use function notation
    \item Use the vertical line test to identify functions
    \item Find the domain and range of functions
    \item Graph linear functions
    \item Find the domain and range of functions graphically
    \item Graph nonlinear functions
    \item Translate and reflect graphs of functions
\end{alphalist}
 
\noindent Chapter 4:
\begin{alphalist}
    \item Read and interpret inequality symbols
    \item Graph intervals and use interval and set-builder notation
    \item Solve linear inequalities using properties of inequality
    \item Use linear inequalities to solve problems
    \item Find the intersection and union of two sets
    \item Solve double linear inequalities
    \item Solve compound inequalities containing the words and or
    \item Graph linear inequalities in two variables
    \item Solve applied problems involving linear inequalities in two variables
    \item Solve systems of linear inequalities
    \item Graph compound inequalities
    \item Solve problems involving systems of linear inequalities
\end{alphalist}
\noindent Chapter 5:
\begin{alphalist}
    \item Identify bases and exponents
    \item Use the exponent rules to simplify expressions
    \item Define and classify polynomials
    \item Evaluate polynomial functions
    \item Find function values and the domain and range of polynomial functions graphically
    \item Add, subtract, and multiply polynomials
    \item Find special products
    \item Use multiplication to simplify expressions
    \item Find the greatest common factor of a list of terms
    \item Factor out the greatest common factor
    \item Factor by grouping
    \item Use factoring to solve formulas for a specific variable
    \item Factor perfect square trinomials
    \item Factor trinomials in the form $x^2 + bx + c$ and $ax^2 + bx + c$
    \item Use substitution to factor trinomials
    \item Use the grouping method to factor trinomials
    \item Factor the difference of two squares
    \item Factor the sum and difference of two cubes
    \item Solve higher-degree polynomial equations by factoring
    \item Use quadratic equations to solve problems
\end{alphalist}
 
\noindent Chapter 6:
\begin{alphalist}
    \item Define rational expressions and functions
    \item Evaluate rational functions
    \item Find the domain of a rational function
    \item Recognize the graphs of rational functions
    \item Simplify rational expressions
    \item Multiply and divide rational expressions
    \item Perform mixed operations on rational expressions
    \item Find the least common denominator of rational expressions
    \item Add and subtract rational expressions
    \item Simplify complex fractions using division
    \item Simplify complex fractions using the LCD
    \item Divide a polynomial by a monomial
    \item Divide a polynomial by a polynomial
    \item Divide polynomials with missing terms 
    \item Solve rational equations
    \item Solve rational equations with extraneous solutions
    \item Solve formulas for a specific variable
    \item Solve shared-work and uniform-motion problems
\end{alphalist}

\noindent Chapter 7:
\begin{alphalist}
    \item Find square, cube and nth roots
    \item Graph the square root and cube root functions
    \item Evaluate radical functions
    \item Convert between radicals and rational exponents
    \item Use rules of exponents to simplify expressions
    \item Simplify radical expressions by using prime factorization, and the product and quotient rule
    \item Add and subtract radical expressions
    \item Rationalize numerators and denominators of radical expressions
    \item Multiply and divide radical expressions
    \item Solve equations containing one and two radicals
    \item Solve formulas containing radicals
    \item Use the Pythagorean Theorem to solve problems 
\end{alphalist}
\noindent Chapter 8:
\begin{alphalist}
    \item Use the square root property to solve quadratic equations
    \item Solve quadratic equations by completing the square
    \item Derive the quadratic formula
    \item Solve quadratic equations using the quadratic formula
    \item Use the quadratic formula to solve application problems
    \item Use the discriminant to determine the number and type of solutions to quadratic equations
    \item Solve application problems using quadratic equations
    \item Find the vertex of a quadratic function using $-b/2a$
    \item Graph quadratic functions
    \item Graph functions of the form $f(x) = ax^2 + bx + c$ by completing the square
    \item Determine the minimum and maximum values of quadratic functions
    \item Solve quadratic equations graphically
    \item Solve quadratic inequalities
    \item Solve rational inequalities
    \item Graph nonlinear inequalities in two variables
\end{alphalist}

\noindent Note: The online MATH  101 course is slightly different. The online course does all of Chapter 1 rather than starting in \S 1.5 and
it eliminates all of chapter 4.  For the online class, the following objectives cover sections \S 1.1 - \S 1.4.
 
\begin{alphalist}
    \item Write verbal and mathematical models
    \item Use equations to construct tables of data
    \item Define the set of natural numbers, whole numbers, integers, rational numbers, irrational numbers, and real numbers.
    \item Graph real numbers
    \item Order the real numbers
    \item Find the opposite and absolute value of real numbers
    \item Add, subtract, multiply, and divide real numbers
    \item Find powers and square roots of real numbers
    \item Use the order of operations rule
    \item Evaluate algebraic expressions
    \item Identify terms, factors, and coefficients
    \item Identify and use properties of real numbers
    \item Simplify algebraic expressions using the properties of real numbers

\end{alphalist}



\subsubsection*{MATH 104  Concepts in Mathematics and Statistics}


On completion of this course, students will be able to:
\begin{alphalist}
    \item Identify, write, and graph linear functions.
    \item Write and graph quadratic functions.
    \item Solve linear and quadratic equations and inequalities.
    \item Solve two-variable systems of linear equations.
    \item Determine probabilities for independent events.
    \item Use the multiplication principle, permutations, and combinations.
    \item Use measures of center including mean, median, and mode.
    \item Use measures of variation including standard deviation, range, and variance.
    \item Describe distributions and create box plots.
\end{alphalist}


\subsubsection*{MATH 202, Calculus II with Analytic Geometry}

 On completion of this course, students will be able to
 \begin{alphalist}
    \item use definite integrals to solve problems involving volume, arc length, 
       surface area, work, and center of mass. 
    \item use integration by parts, trigonometric substitution, and partial fractions 
        to evaluate definite and indefinite integrals.
    \item apply the concepts of limits, convergence, and divergence to evaluate 
        improper integrals.
    \item determine convergence or divergence of sequences and series.
    \item use Taylor and MacLaurin series to represent functions and integrate 
        functions.
    \item use parametrizations and polar coordinates to find areas and arc lengths.
 \end{alphalist}

\subsubsection*{MATH 250, Foundations of Math}


On completion of this course, students will
\begin{alphalist}
    \item gain an understanding of naïve set theory. 
    \item gain an understanding of symbolic logic, quantifiers, and functions.
    \item gain an understanding of direct proofs, proofs by contrapositive, and proofs by induction.
    \item gain the ability to read and understand mathematical proofs.
    \item gain the problem solving skills that needed to create a mathematical proof.
\end{alphalist}

\subsubsection*{MATH 251, Inquiry and Proof in 9-12 Mathematics}


On completion of this course, students will be able to:
\begin{alphalist}
\item Articulate and utilize mathematical practices essential for 9-12 mathematics.  
\item Articulate the roles proof can play in secondary mathematics instruction.
\item Engage in mathematical inquiry using technological and mathematical tools.
\item Articulate mathematical arguments with precise mathematical language and symbols.
\item Communicate technical mathematical justifications in a manner appropriate for secondary students.
\item Determine how essential understanding of proof is embedded across mathematical content.
\end{alphalist}

\subsubsection*{MATH 260, Calculus III}


On completion of this course, students will 
\begin{alphalist} 
    \item be able to use vectors to solve geometric problems and basic engineering statics problems.
    \item understand the concept of partial derivatives and limits of multi-variable functions.
    \item understand the concept of a parameterized curve and understand the concept of curvature.
    \item will be able to use partial derivatives to solve multi-variable optimization problems.
    \item will understand the concept of a line integral and apply it to basic physics problems involving energy and work.
    \item be able to set up and evaluate multiple integrals using Cartesian, polar, and spherical coordinates to find volumes, 
    surface areas, centroids, and moments of inertia.
    \item understand the concepts of the vector divergence, gradient, and curl in 
    three-dimensional space.
    \item understand the Green theorem and the Divergence theorem and be able to 
    apply these concepts to problems involving surface and line integrals.
\end{alphalist}

\subsubsection*{MATH 270, Methods in Middle and High School Mathematics Teaching I}

On completion of this course, students will be able to:
\begin{alphalist}

\item Thoroughly describe what is meant by “doing,”  “teaching,”  and “learning” mathematics in their own words and with the support of mathematics educational research.
\item Explain and provide real-life examples of the eight research-based mathematics teaching practices.
\item Identify and begin creating opportunities for high-quality instruction that includes mathematical 
discourse, productive struggle, purposeful questioning, and the connecting of multiple representations. 
\item Articulate the essential mathematical concepts of 6-12 mathematics curriculum regarding number, algebra/functions, statistics/probability, and geometry/measurement.
\item Explain the organization and benefits of the NCTM Standards, the Common Core State Standards, Nebraska State Standards, 
\item Explain the history and current trends in mathematics education.
\item Define NCTM and NATM, explain membership benefits associated with each organization, and articulate the importance of professional affiliations.
\item Build upon foundational understanding of mathematics education and research-based mathematics teaching.
\item Be reflective of your own learning and realize how his/her own understanding influences student learning.
\end{alphalist}

\subsubsection*{MATH 271, Field Experience in Middle and High School Mathematics I}

On completion of this course, students will be able to:
\begin{alphalist}
\item Identify research-based mathematics teaching practices that are included in the classroom, as well as how they could be incorporated.
\item Engage 6-12 students in developmentally appropriate mathematical activities. 
\item Work with a diverse range of students individually, in small groups, and in large class settings.
\item Plan, facilitate, and reflect upon mathematical tasks that promote reasoning and sense making.
\item Collect and analyze data to determine if 6-12 students have built new knowledge.
\item Meet expectations of all Teacher Education Dispositions, including:
\begin{betalist}
    \item Demonstrate effective oral communication skills
    \item Demonstrate effective written communication skills
    \item Demonstrate professionalism
    \item Demonstrate a positive and enthusiastic attitude
    \item Demonstrate preparedness in teaching and learning
    \item Exhibits an appreciation of and value for cultural and academic diversity
    \item Collaborates effectively with stakeholders
    \item Demonstrates self-regulated learner behaviors/takes initiative
    \item Exhibits the social and emotional intelligence to promote personal and educational goals/stability
\end{betalist}
\end{alphalist}

\subsubsection*{MATH 280, Linear Algebra}

\subsubsection*{MATH 300, Tutoring in Mathematics}

\subsubsection*{MATH 305, Differential Equations}

On completion of this class, students will: 
\begin{alphalist}
    \item Know the classical methods for solving first order differential equations (ODEs).
    \item Be able to set up and solve applied problems involving first order ODEs.
    \item Know the classical methods for solving second order ODEs.
    \item Know the classical methods for solving systems of first order ODEs.
    \item Be able to solve initial value problems for first and second order ODEs and for systems of ODEs.
\end{alphalist}

\subsubsection*{MATH 310, College Geometry}

On completion of this course, students will be able to
\begin{alphalist}
    \item understand the basic definitions, axioms, and important theorems in neutral geometry.
    \item compare Euclidean geometry, hyperbolic geometry, and elliptical geometry.
    \item use the axiomatic or transformational approach to prove theorems in neutral/Euclidean geometry.
    \item use interactive geometry software for constructions in plane geometry, and
    explain geometric concepts and results in concrete models.
\end{alphalist}

\subsubsection*{MATH 330, Math for Elementary Teachers II}

On completion of this course, students will be able to: 
\begin{alphalist}
\item Explain and perform operations with fractions and decimals. 
\item Apply understanding of ratios, percentages, and proportions to real-life situations. 
\item Identify, categorize, compare and contrast various shapes and solids. 
\item Determine the area, surface area, and volume of two and three-dimensional objects. 
\item Approach mathematics problems using a variety of methods. 
\item Explain mathematical concepts to students at their level of understanding.
\end{alphalist}

\subsubsection*{MATH 350,  Abstract Algebra}

\subsubsection*{MATH 365, Complex Analysis}

Upon completion of this course, students will 
\begin{alphalist}
    \item be able to represent complex numbers algebraically and geometrically.
    \item be able to use the definition of the limit to prove that a function has a limit.
    \item be able to find derivatives of complex valued functions from the limit definition.
    \item understand the connection between the complex exponential and the trigonometric functions and be able to 
        prove trigonometric identities using these connections.
    \item be able to use the Cauchy-Riemann equations to determine if a function is 
          analytic.
    \item be able to represent functions using Laurent and power series and be able 
           to find function residues, poles, and pole order.
    \item be able to evaluate contour integrals using residues.
    \item understand the Cauchy integral formula and its consequences,  including the 
       fundamental theorem of algebra.
\end{alphalist}


\subsubsection*{MATH 390, Research Experience in Mathematics}


\subsubsection*{MATH 399, Internship}


\subsubsection*{MATH 400, History of Mathematics}

On completion of this course, students will 
\begin{alphalist}
    \item understand the progression of mathematics through history.
    \item understand the history of a variety of mathematical topics.
    \item gain an understanding of several mathematicians through history.
\end{alphalist}

\subsubsection*{MATH 404, Theory of Numbers}

On completion of this course, students will be able to:
\begin{alphalist}
    \item explain the concepts of divisibility, prime number, and congruence. 
    \item calculate the great common divisor using the Euclidean algorithm and prime factorization.
    \item solve linear congruence and quadratic congruence.
    \item understand Wilson's Theorem and Fermat's Little Theorem. 
    \item compute Euler's torsion function and other important multiplicative functions.
    \item use primitive roots and index arithmetic to solve higher-order congruence.  
    \item solve linear Diophantine equations and find primitive Pythagorean triples.
    \item understand how rational numbers are related to repeating decimals and continued fractions.
\end{alphalist}

\subsubsection*{MATH 413, Discrete Mathematics}

On completion of MATH 413, students will 
\begin{alphalist}
    \item gain an understanding of counting principles and how to apply them.
    \item gain an understanding of discrete structures and how to use and analyze them, including induction, recursion, and probabilistic methods.
\end{alphalist}

\subsubsection*{MATH 420, Numerical Analysis}

On completion of this course, students will
\begin{alphalist}
    \item understand IEEE arithmetic and know the rules for accurate computation.
    \item be able to determine the time complexity of algorithms.
    \item understand the concepts of linear and quadratic convergence and use these concepts to analyze the efficiency of an algorithm.
    \item develop an understanding of the algorithms for solving linear and nonlinear equations, interpolation, 
       quadrature, and solution of differential equations.
    \item be able to use a programming language and graphical tools to solve problems numerically.
\end{alphalist}

\subsubsection*{MATH 430,  Middle School Mathematics}

Upon completion of this course, students will be able to: 
\begin{alphalist}
\item Conceptualize the real number system, including rational and irrational numbers. 
\item Explain algebraic procedures (i.e. solving equations/inequalities, laws of exponents). 
\item Simplify exponential and radical expressions. 
\item Apply transformation properties of congruent and similar figures. 
\item Identify and interpret graphs and functions. 
\item Compare and contrast different types of functions. 
\item Apply the Pythagorean Theorem to variety of situations. 
\item Use the coordinate plane to solve problems and display mathematics. 
\item Calculate, display and interpret statistical measures. 
\item Perform probability simulations and interpret results. 
\item Approach mathematical problems using a variety of methods. 
\item Use various teaching models and techniques of curriculum delivery including effective questioning, cooperative learning, inquiry, and constructivist learning. 
\item Explain mathematical concepts to students at their level of understanding. 
\item Be reflective of own learning and realize how his/her own understanding influences student learning.
\end{alphalist}

\subsubsection*{MATH 445, Actuarial Science Seminar}


\subsubsection*{MATH 460, Advanced Calculus I}

On completion of this class, students will
\begin{alphalist}
    \item be able to prove basic propositions that involve the fundamentals of point set topology, including the concepts of open sets, closed sets, boundary points, and limit points.
     \item be able to prove basic propositions that involve the concept of the infimum and supremum. 
    \item demonstrate competence with basic properties of sequences including determining convergence and proving results involving the sum, difference, product, and quotient of sequences.
    \item be able to use the definitions of continuity, the limit, and the derivative to prove basic propositions involving these concepts as well as be able to prove facts about specific functions.
    \item demonstrate the ability to use the Mean Value Theorem to prove theorems. 
    \item be able to define and evaluate the lower, upper, and general Riemann sums.
    \item demonstrate a solid understanding of the fundamental theorem of calculus.
\end{alphalist}

\subsubsection*{MATH 465, Advanced Study in 9-12 Mathematics}

On  completion of this course, students will be able to:
\begin{alphalist}

\item Connect higher level content knowledge to essential content in secondary mathematics.
\item Explain the impact of higher-level mathematical content knowledge on their teaching of high school students.
\item Articulate and utilize mathematical practices essential for 9-12 mathematics.
\item Describe essential understandings for 9-12 students in number theory, algebra/functions, statistics, probability, and calculus.
\item Demonstrate the interconnectedness of mathematics among mathematical ideas.
\item Utilize technological tools to explore essential mathematical content in number/quantity, algebra, statistics/probability, and calculus.
\end{alphalist}

\subsubsection*{MATH 470, Methods in Middle and High School Mathematics Teaching II}

On completion of this course, students will be able to:
\begin{alphalist}
\item Develop effective unit and lessons that support district, state and national standards and are developmentally appropriate.
\item Incorporate various forms of communication and connections (including within the subject area, to other disciplines and to real life) into lessons.
\item Diagnose and assess student performance in a variety of ways, including formative, summative, open ended and performance assessments.
\item Address student diversity and various learning needs in lessons and units.
\item Use teaching methods and techniques of curriculum delivery including effective questioning, cooperative learning, inquiry, technology and problem solving
\item Incorporate various classroom organization and management techniques when teaching students.
\item Develop mathematical experiences for students, which will lead to positive dispositions toward math.
 \item Explain the organization and benefits of the NCTM Standards, the Common Core State Standards, Nebraska State Standards, and current trends in mathematics education.
\item Define NCTM and NATM, and explain membership benefits associated with each organization, and articulate the importance of professional affiliations.
\item Be reflective of own learning and realize how his/her own understanding influences student learning.
\end{alphalist}

\subsubsection*{MATH 471, Field Experience in Middle and High School Mathematics II}

On completion of this course, students will:
\begin{alphalist}
\item Utilize research-based mathematics teaching practices in the classroom.
\item Engage 6-12 students in developmentally appropriate mathematics lessons.
\item Incorporate technology and tools into the 6-12 class in order to enhance mathematical understanding.
\item Work with a diverse range of students individually, in small groups, and in large class settings.
\item Plan, facilitate, and reflect upon mathematics lessons that promote reasoning and sense making.
\item Collect and analyze data to determine if 6-12 students have built new knowledge. 
\item Collaborate with colleagues, other school professionals, families, and stakeholders.
\item Continue to develop as a reflective practitioner.
\item Meet expectations of all Teacher Education Dispositions, including:
\begin{betalist}
\item Demonstrate effective oral communication skills
\item Demonstrate effective written communication skills
\item Demonstrate professionalism
\item Demonstrate a positive and enthusiastic attitude
\item Demonstrate preparedness in teaching and learning
\item Exhibits an appreciation of and value for cultural and academic diversity
\item Collaborates effectively with stakeholders
\item Demonstrates self-regulated learner behaviors/takes initiative
\item Exhibits the social and emotional intelligence to promote personal and educational goals/stability
\end{betalist}
\end{alphalist}

\subsubsection*{MATH 490, Special Topics in Mathematics}

The learning outcomes for MATH 490 vary by the course content.

\subsubsection*{MATH 495, Independent Study in Mathematics}


The learning outcomes for MATH 495 vary by the course content.

\subsubsection*{MATH 496, Mathematics Seminar}

The learning outcomes for MATH 496 vary by the course content.


\subsubsection*{STAT 345, Applied Statistics I}

The class STAT 345 has not been offered in over ten years
and it is not required by any UNK degree program. The
department does not have a syllabus for this  course.

\subsubsection*{STAT 399, Internship}

\subsubsection*{STAT 441, Probability and Statistics}


\subsubsection*{STAT 442, Mathematical Statistics}

The class STAT 442 has not been offered in over ten years
and it is not required by any UNK degree program. The
department does not have a syllabus for this course.

\subsubsection*{STAT 495, Independent Study in Statistics}

The learning outcomes for STAT 495 vary by the course content.

%\end{multicols}
\end{document}


