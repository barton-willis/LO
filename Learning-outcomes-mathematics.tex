
\begin{flushleft}
\textbf{Learning Outcomes for Mathematics and Statistics Classes} \\
\vspace{0.1in}
\emph{Approved by the Department of Mathematics and Statistics on XXX}
\end{flushleft}

\subsubsection*{Learning Outcomes for General Studies Mathematics & Statistics Classes (Loper 4)}

Our General Studies classes are \textbf{MATH- 102} (College Algebra),
\textbf{MATH 103} (Plane Trigonometry),
\textbf{MATH 106} (Mathematics for Liberal Arts),
\textbf{MATH 115} (Calculus I with Analytic Geometry),
\textbf{MATH 120} (Finite Mathematics),
\textbf{MATH 123} (Applied Calculus I),
\textbf{MATH 230} (Math for Elementary Teachers),
\textbf{STAT 235} Introduction to  Statistics for  Social Sciences),
and \textbf{STAT 241}  Elementary Statistics)

The learning outcomes for these classes are:

\begin{alphalist}
\item Can describe problems using mathematical, statistical, or programming language
\item Can solve problems using mathematical, statistical, or programming techniques
\item Can construct logical arguments using mathematical, statistical, or programming concepts
\item Can interpret and express numerical data or graphical information using mathematical, statistical, or programming concepts and methods
\end{alphalist}

Learning Outcomes for undergraduate mathematics classes
\subsubsection{MATH 90, Elementary Algebra}

Learning Outcomes:

Chapter 1:
\begin{alphalist}
    \item Evaluate algebraic expressions
    \item Translate English phrases/sentences into algebraic expressions
    \item Determine whether a number is a solution of an equation
    \item Evaluate formulas
    \item Convert between mixed numbers and improper fractions
    \item Write the prime factorization of a composite number
    \item Reduce or simplify fractions
    \item Add, subtract, multiply and divide fractions
    \item Solve problems involving fractions in algebra
    \item Define the sets that make up the real numbers
    \item Graph numbers on a number line
    \item Express rational numbers as decimals
    \item Classify numbers as belonging to one or more sets of the real numbers
    \item Understand and use inequality symbols
    \item Find the absolute value of a real number
    \item Understand and use the vocabulary of algebraic expressions
    \item Use the commutative, associate, and distributive properties
    \item Simplify algebraic expressions
    \item Understand and use the vocabulary of algebraic expressions
    \item Add numbers with or without a number line
    \item Add, subtract, multiply and divide real numbers
    \item Use identity and inverse properties for addition and multiplication 
    \item Use the basic operations to simplify algebraic expressions 
    \item Solve applied problems using a series of basic operations
    \item Evaluate/simplify exponential expressions
    \item Use the order of operations agreement
    \item Evaluate mathematical models
\end{alphalist}
Chapter 2:
 
\begin{alphalist}
    \item Identify linear equations in one variable
    \item Use the addition and multiplication properties of equality to solve equations
    \item Solve linear equations
    \item Solve linear equations involving fractions and decimals
    \item Identify equations with no solution or infinitely many solutions
    \item Solve applied problems using formulas
    \item Solve a formula for a variable
    \item Use the percent formula
    \item Solve applied problems involving percent change
    \item Translate English phrases into algebraic expressions
    \item Solve algebraic word problems using linear equations
    \item Solve problems using formulas for perimeter, the circumference of a circle, area, and volume
    \item Solve problems involving the angles of a triangle
    \item Solve problems involving complementary and supplementary angles
\end{alphalist}
Chapter 3:
 \begin{alphalist}
    \item Plot/find coordinates of points in the rectangular coordinate system
    \item Determine whether an ordered pair is a solution of an equation
    \item Find solutions of an equation in two variables
    \item Use point plotting to graph linear equations
    \item Use graphs of linear equations to solve problems
    \item Use a graph to identify intercepts
    \item Graph a linear equation in two variables using intercepts
    \item Graph horizontal or vertical lines
    \item Compute the slope of a line
    \item Use slope to show that lines are parallel or perpendicular
    \item Calculate rate of change in applied situations
    \item Find the slope of a line and its y-intercept from its equation
    \item Graph lines in slope-intercept form
    \item Use slope and the y-intercept to graph $Ax + By = C$
    \item Use the slope and y-intercept to model data
    \item Use the point-slope form to write equations of a line
    \item Write linear equations that model data and make predictions
 \end{alphalist}
Chapter 5:
 \begin{alphalist}
    \item Understand the vocabulary used to describe polynomials
    \item Add and subtract polynomials
    \item Graph equations defined by polynomials of degree 2
    \item Use FOIL in polynomial multiplication
    \item Find the square of a binomial sum or difference
    \item Multiply polynomials
    \item Add, subtract, and multiply polynomials in several variables
    \item Evaluate polynomials in several variables
    \item Divide monomials
    \item Check polynomial division
    \item Divide a polynomial by a monomial 
    \item Divide polynomials by binomials
    \item Use the negative exponent rule
    \item Simplify exponential expressions
 \end{alphalist}
Chapter 6:
 \begin{alphalist}
    \item Find the greatest common factor and factor it out of a polynomial
    \item Factor by grouping
    \item Factor trinomials of the form $x^2 + bx + c$
    \item Factor trinomials by trial and error
    \item Factor trinomials by grouping
    \item Factor the difference of two squares
    \item Factor perfect square trinomials
    \item Factor the sum or difference of two cubes
    \item Use a general strategy to recognize the appropriate method of factoring a polynomial
    \item Solve quadratic equations by factoring
    \item Solve problems using quadratic equations
 \end{alphalist}
Chapter 7:
 \begin{alphalist}
    \item Find numbers for which a rational expression is undefined
    \item Simplify rational expressions
    \item Solve applied problems involving rational expressions
    \item Multiply and divide rational expressions
    \item Add and subtract rational expressions
    \item Simplify complex rational expressions by dividing
    \item Simplify complex rational expressions by multiplying by the LCD
    \item Solve rational equations
    \item Solve problems involving formulas with rational expressions
 \end{alphalist}
Chapter 9:
 \begin{alphalist}
    \item Solve quadratic equations using the square root property
    \item Solve problems using the Pythagorean Theorem
    \item Find the distance between two points
    \item Complete the square of a binomial
    \item Solve quadratic equations by completing the square
    \item Solve quadratic equations using the quadratic formula
    \item Solve problems using quadratic equations
 \end{alphalist}


MATH 101 – Intermediate Algebra 3 credit hours
The course which includes a study of the properties of real numbers, polynomials, fundamental operations, factoring, exponents, and radicals, linear and quadratic equations, and other selected topics, all of which are necessary for the study of college algebra. Not a General Studies course.
Prerequisite: MATH 090 or Math ACT Score of 17 or greater and one year of high school algebra Students may not enroll in MATH 101 after earning credit for any General Studies Mathematics class.

\subsubsection{Learning Outcomes for MATH 101 (Intermediate Algebra)}


Chapter 1:
\begin{alphalist}
    \item Determine whether a number is a solution to a linear equation
    \item Solve linear equations using the properties of equality
    \item Identify identities and contradictions
    \item Use geometric formulas to find the perimeter of plane two-dimensional figures
    \item Find the circumference of a circle
    \item Use formulas to find the area and volume of various geometric figures
    \item Solve equations for a specific variable
    \item Solve application problems using formulas
    \item Use problem solving techniques to solve percent, investment, uniform motion, and mixture problems.
\end{alphalist}
Chapter 2:
\end{alphalist}
    \item Plot ordered pairs and determine the coordinates of a point
    \item Read graphs and graph paired data
    \item Find the midpoint of a line segment
    \item Determine whether an ordered pair is a solution of an equation
    \item Graph linear equations in two variables
    \item Use linear models to solve applied problems
    \item Calculate an average rate of change
    \item Find the slope of a line using a graph or the slope formula
    \item Use slope to solve application problems
    \item Determine whether lines are parallel or perpendicular using slope
    \item Use the slope-intercept and point-slope forms to write equations of lines
    \item Identify functions and use function notation
    \item Use the vertical line test to identify functions
    \item Find the domain and range of functions
    \item Graph linear functions
    \item Find the domain and range of functions graphically
    \item Graph nonlinear functions
    \item Translate and reflect graphs of functions
\end{alphalist}
 
Chapter 4:
\begin{alphalist}
    \item Read and interpret inequality symbols
    \item Graph intervals and use interval and set-builder notation
    \item Solve linear inequalities using properties of inequality
    \item Use linear inequalities to solve problems
    \item Find the intersection and union of two sets
    \item Solve double linear inequalities
    \item Solve compound inequalities containing the words and or
    \item Graph linear inequalities in two variables
    \item Solve applied problems involving linear inequalities in two variables
    \item Solve systems of linear inequalities
    \item Graph compound inequalities
    \item Solve problems involving systems of linear inequalities
\end{alphalist}
Chapter 5:
\begin{alphalist}
    \item Identify bases and exponents
    \item Use the exponent rules to simplify expressions
    \item Define and classify polynomials
    \item Evaluate polynomial functions
    \item Find function values and the domain and range of polynomial functions graphically
    \item Add, subtract, and multiply polynomials
    \item Find special products
    \item Use multiplication to simplify expressions
    \item Find the greatest common factor of a list of terms
    \item Factor out the greatest common factor
    \item Factor by grouping
    \item Use factoring to solve formulas for a specific variable
    \item Factor perfect square trinomials
    \item Factor trinomials in the form x^2 + bx + c and ax^2 + bx + c
    \item Use substitution to factor trinomials
    \item Use the grouping method to factor trinomials
    \item Factor the difference of two squares
    \item Factor the sum and difference of two cubes
    \item Solve higher-degree polynomial equations by factoring
    \item Use quadratic equations to solve problems
\end{alphalist}
 
Chapter 6:
\begin{alphalist}
    \item Define rational expressions and functions
    \item Evaluate rational functions
    \item Find the domain of a rational function
    \item Recognize the graphs of rational functions
    \item Simplify rational expressions
    \item Multiply and divide rational expressions
    \item Perform mixed operations on rational expressions
    \item Find the least common denominator of rational expressions
    \item Add and subtract rational expressions
    \item Simplify complex fractions using division
    \item Simplify complex fractions using the LCD
    \item Divide a polynomial by a monomial
    \item Divide a polynomial by a polynomial
    \item Divide polynomials with missing terms 
    \item Solve rational equations
    \item Solve rational equations with extraneous solutions
    \item Solve formulas for a specific variable
    \item Solve shared-work and uniform-motion problems
\end{alphalist}

Chapter 7:
\begin{alphalist}
    \item Find square, cube and nth roots
    \item Graph the square root and cube root functions
    \item Evaluate radical functions
    \item Convert between radicals and rational exponents
    \item Use rules of exponents to simplify expressions
    \item Simplify radical expressions by using prime factorization, and the product and quotient rule
    \item Add and subtract radical expressions
    \item Rationalize numerators and denominators of radical expressions
    \item Multiply and divide radical expressions
    \item Solve equations containing one and two radicals
    \item Solve formulas containing radicals
    \item Use the Pythagorean Theorem to solve problems 
\end{alphalist}
Chapter 8:
\begin{alphalist}
    \item Use the square root property to solve quadratic equations
    \item Solve quadratic equations by completing the square
    \item Derive the quadratic formula
    \item Solve quadratic equations using the quadratic formula
    \item Use the quadratic formula to solve application problems
    \item Use the discriminant to determine the number and type of solutions to quadratic equations
    \item Solve application problems using quadratic equations
    \item Find the vertex of a quadratic function using -b/2a
    \item Graph quadratic functions
    \item Graph functions of the form f(x) = ax^2 + bx + c by completing the square
    \item Determine the minimum and maximum values of quadratic functions
    \item Solve quadratic equations graphically
    \item Solve quadratic inequalities
    \item Solve rational inequalities
    \item Graph nonlinear inequalities in two variables
\end{alphalist}

Note: The online Math 101 course that was developed by the team is slightly different.
It does all of chapter 1 rather than starting in section 1.5 as I have done here and eliminates all of chapter 4. 
The following objectives cover sections \S 1.1 - \S 1.4.
 
\begin{alphalist}
    \item Write verbal and mathematical models
    \item Use equations to construct tables of data
    \item Define the set of natural numbers, whole numbers, integers, rational numbers, irrational numbers, and real numbers.
    \item Graph real numbers
    \item Order the real numbers
    \item Find the opposite and absolute value of real numbers
    \item Add, subtract, multiply, and divide real numbers
    \item Find powers and square roots of real numbers
    \item Use the order of operations rule
    \item Evaluate algebraic expressions
    \item Identify terms, factors, and coefficients
    \item Identify and use properties of real numbers
    \item Simplify algebraic expressions using the properties of real numbers

\end{alphalist}



\subsubsection{MATH 104  Concepts in Mathematics and Statistic}


Upon completion of this course, students will be able to:
\begin{alphalist}
    \item Identify, write, and graph linear functions.
    \item Write, and graph quadratic functions.
    \item Solve linear and quadratic equations and inequalities.
    \item Solve two-variable systems of linear equations.
    \item Determine probabilities for independent events.
    \item Use the multiplication principle, permutations, and combinations.
    \item Use measures of center including mean, median, and mode.
    \item Use measures of variation including standard deviation, range, and variance.
    \item Describe distributions and create box plots.
\end{alphalist}


MATH 202 – Calculus II with Analytic Geometry 5 credit hours
A continuation of MATH 115 including the differentiation and integration of transcendental functions, methods of formal integration with applications, series.
Prerequisite: MATH 115 or Math ACT score of 25 or greater and one year of high school calculus.

Learning Outcomes MATH 202
 Upon successful completion of this course, students will be able to 
    \item use definite integrals to solve problems involving volume, arc length, surface area, work, and center of mass, 
    \item use integration by parts, trigonometric substitution, partial fractions to evaluate definite and indefinite integrals,
    \item apply the concepts of limits, convergence, and divergence to evaluate some classes of improper integrals,
    \item determine convergence or divergence of sequences and series,
    \item use Taylor and MacLaurin series to represent functions and integrate functions,
    \item use parametrizations and polar coordinates to find areas and arc lengths.


MATH 230 – Math for Elementary Teachers I 3 credit hours
In this course, preservice teachers develop knowledge of mathematics important for the effective teaching of PK-6 students. The mathematical topics investigated in the course include problem solving, the number system, alternate base systems, operations with whole numbers and integers, introductory number theory concepts, and data analysis. In all of these topics, preservice teachers learn to develop appropriate mathematical explanations, understand student reasoning about mathematics, and communicate mathematical reasoning.
Prerequisite: MATH 102 or MATH 104 or Math ACT score of 20 or greater and four years of high school mathematics including two years of algebra and one year of geometry and a senior level mathematics course.

Learning Outcomes for MATH 230 (Loper 4)
a. Can describe problems using mathematical, statistical, or programming language  
b. Can solve problems using mathematical, statistical, or programming techniques  
c. Can construct logical arguments using mathematical, statistical, or programming concepts  
d. Can interpret and express numerical data or graphical information using mathematical, statistical, or programming concepts and methods  

MATH 250 – Foundations of Math 3 credit hours
Topics of sets and symbolic logic are studied with the objective of using them in the detailed study of the nature of different types of proofs used in mathematics. Also, the processes of problem solving are studied for developing strategies of problem solving.
Prerequisite: MATH 115 or MATH 123
Learning Outcomes for MATH 250

On completion of MATH 250, students will 
    \item gain an understanding of naïve set theory. 
    \item gain an understanding of symbolic logic, quantifiers, and functions.
    \item gain an understanding of direct proofs, proofs by contrapositive, and proofs by induction.
    \item gain the ability to read and understand mathematical proofs.
    \item gain the problem-solving skills that needed to create a mathematical proof.
MATH 251 – Inquiry and Proof in 9-12 Mathematics     1 credit hour
This course is an introduction to the 9-12 mathematics curriculum with a focus on the role of mathematical inquiry and justification in the form of proof. Preservice teachers will be introduced to applications and the role of mathematical proofs in high school curriculum. Students will also engage in the process of mathematical inquiry leading to proof in a manner applicable to secondary teaching.
Prerequisite: MATH 115
Learning Outcomes for MATH 251
Upon completion of this course, students will be able to:
\item Articulate and utilize mathematical practices essential for 9-12 mathematics.  
\item Articulate the roles proof can play in secondary mathematics instruction.
\item Engage in mathematical inquiry using technological and mathematical tools.
\item Articulate mathematical arguments with precise mathematical language and symbols.
\item Communicate technical mathematical justifications in a manner appropriate for secondary students.
\item Determine how essential understanding of proof is embedded across mathematical content.

MATH 260 – Calculus III 5 credit hours
A continuation of MATH 202. Vector calculus, partial derivatives and multiple integrals.
Department Consent Required
Prerequisite: MATH 202 or equivalent preparation

Learning Outcomes for MATH 260
Upon completion, 
    \item students will be able to use vectors to solve geometric problems and basic engineering statics problems.
    \item students will understand the concept of partial derivatives and limits of multi-variable functions.
    \item students will understand the concept of a parameterized curve and understand the concept of curvature.
    \item students will be able to use partial derivatives to solve multi-variable optimization problems.
    \item students will understand the concept of a line integral and apply to basic physics problems involving energy and work.
    \item students will be able to set up and evaluate multiple integrals using Cartesian, polar, and spherical coordinates to find volumes, surface areas, centroids, and moments of inertia.
    \item students will understand the concepts of the vector divergence, gradient, and curl in three-dimensional space.
    \item students will understand the Green theorem and the Divergence theorem and be able to apply these concepts to problems involving surface and line integrals.
MATH 270 – Methods in Middle and High School Mathematics Teaching I     2 credit hours
In this initial methods course, preservice teachers develop a foundational understanding of pedagogy specific to 6-12 grade mathematics teaching. The topics investigated in the course include mathematics instructional methodology, research-based math teaching practices, mathematics standards, mathematics curricula, equitable structuring of middle and high school classrooms, and the essential concepts in middle and high school mathematics. In addition, preservice teachers cultivate a strong understanding of the historical and current trends in mathematics education. MATH 271, a co-requisite course, provides the opportunity to identify and put learning into practice.
Prerequisite: TE 100. 
Corequisite: MATH 271.

Learning Outcomes, MATH 270

Upon completion of this course, students will be able to:
\itemThoroughly describe what is meant by “doing,”  “teaching,”  and “learning” mathematics in their own words and with the support of mathematics educational research.
\itemExplain and provide real-life examples of the eight research-based mathematics teaching practices.
\itemIdentify and begin creating opportunities for high-quality instruction that includes mathematical discourse, productive struggle, purposeful questioning, and the connecting of multiple representations. 
\itemArticulate the essential mathematical concepts of 6-12 mathematics curriculum regarding number, algebra/functions, statistics/probability, and geometry/measurement.
\itemExplain the organization and benefits of the NCTM Standards, the Common Core State Standards, Nebraska State Standards, 
\itemExplain the history and current trends in mathematics education.
\itemDefine NCTM and NATM, explain membership benefits associated with each organization, and articulate the importance of professional affiliations.
\itemBuild upon foundational understanding of mathematics education and research-based mathematics teaching.
\itemBe reflective of your own learning and realize how his/her own understanding influences student learning.

MATH 271 – Field Experience in Middle and High School Mathematics I     1 credit hour
This 50 clock-hour mathematics specific field-based experience is designed to introduce students to classroom teaching. Under the mentorship of a practicing 6-12 mathematics teacher and the supervision of a UNK mathematics educator, preservice teachers will actively engage in the teaching of mathematics to 6-12 students.
Prerequisite: TE 100. 
Corequisite: MATH 270.

Learning Outcomes, MATH 271
Students will:
\itemIdentify research-based mathematics teaching practices that are included in the classroom, as well as how they could be incorporated
\itemEngage 6-12 students in developmentally appropriate mathematical activities 
\itemWork with a diverse range of students individually, in small groups, and in large class settings
\itemPlan, facilitate, and reflect upon mathematical tasks that promote reasoning and sense making
\itemCollect and analyze data to determine if 6-12 students have built new knowledge 
\itemMeet expectations of all Teacher Education Dispositions, including:
oDemonstrate effective oral communication skills
oDemonstrate effective written communication skills
oDemonstrate professionalism
oDemonstrate a positive and enthusiastic attitude
oDemonstrate preparedness in teaching and learning
oExhibits an appreciation of and value for cultural and academic diversity
oCollaborates effectively with stakeholders
oDemonstrates self-regulated learner behaviors/takes initiative
oExhibits the social and emotional intelligence to promote personal and educational goals/stability
MATH 280 – Linear Algebra 3 credit hours
Vector spaces, linear transformations, matrices, and determinants.
Prerequisite: MATH 115 or MATH 202 or MATH 260
MATH 300 – Tutoring in Mathematics     1 credit hour
This course provides opportunities for students to reflect on their experience in mathematics subject tutoring and meet the Experiential Learning requirement for degrees in mathematics. Concurrent employment as a Mathematics Subject Tutor in the Learning Commons (LC) within one of the first three semesters of tutoring employment with the LC is required for enrollment in this course.
Department Consent Required
MATH 305 – Differential Equations     3 credit hours
Methods of solution and applications of common types of differential equations.
Prerequisite: MATH 260

Learning Outcomes for MATH 305 (Differential Equations)

On completion of MATH 305, students will 
    \item Know the classical methods for solving first order differential equations (ODEs).
    \item Be able to Set up and solve applied problems involving 1st order ODEs.
    \item Know the classical methods for solving second order ODEs.
    \item Know the classical methods for solving systems of first order ODEs.
    \item Be able to solve initial value problems for first and second order ODEs and for systems of ODEs.

MATH 310 – College Geometry     3 credit hours
Mathematical systems and re-examination of Euclidean geometry from an advanced viewpoint.
Prerequisite: MATH 250

Learning Outcomes for MATH 310
MATH 310 Learning Outcomes: Upon successful completion of this course, students will be able to
    \item understand the basic definitions, axioms, and important theorems in neutral geometry,
    \item compare Euclidean geometry, hyperbolic geometry, and elliptical geometry,
    \item use the axiomatic or transformational approach to prove theorems in neutral/Euclidean geometry,
    \item use interactive geometry software for constructions in plane geometry, and
    \item explain geometric concepts and results in concrete models.

MATH 330 – Math for Elementary Teachers II 3 credit hours
In this course, preservice teachers further develop knowledge of mathematics important for the effective teaching of PK-6 students. The mathematical topics investigated in the course include operations with rational numbers (e.g., fractions and decimals), proportional reasoning (e.g., percents, ratios), two-dimensional and three-dimensional geometric figures, and measurement (e.g., length, area, volume, angles). In all of these topics, preservice teachers learn to develop appropriate mathematical explanations, understand student reasoning about mathematics, and communicate mathematical reasoning.
Prerequisite: MATH 230
Learning Outcomes MATH 330 (Math for Elementary Teachers II)
Upon completion of this course, students will be able to: 
\item Explain and perform operations with fractions and decimals. 
\item Apply understanding of ratios, percentages, and proportions to real-life situations. 
\item Identify, categorize, compare and contrast various shapes and solids. 
\item Determine the area, surface area, and volume of two- and three-dimensional objects. 
\item Approach mathematics problems using a variety of methods. 
\item Explain mathematical concepts to students at their level of understanding.
MATH 350 – Abstract Algebra 3 credit hours
An introduction to modern algebra, including a brief study of groups, rings, integral domains and fields. Prerequisite: MATH 250 or permission of instructor.
MATH 365 – Complex Analysis 3 credit hours
Complex analysis is an introduction to the theory of complex variables and the calculus of analytic functions. Topics covered include the calculus of residues, the Cauchy Integration theorem, and the extension of exponential, logarithmic, and trigonometric functions to the complex plane.
Prerequisite: MATH 260

Learning Outcomes MATH 365

Upon completion, students will 
    \item  be able to represent complex numbers algebraically and geometrically.
    \item be able to use the definition of the limit to prove that a function has a limit.
    \item be able to find derivatives of complex valued functions from the limit definition.
    \item understand the connection between the complex exponential and the trigonometric functions and be able to prove trigonometric identities using these connections.
    \item be able to use the Cauchy-Riemann equations to determine if a function is analyic.
    \item be able to represent functions using Laurent series and power series and be able to find function residues, poles, and pole order.
    \item be able to evaluate contour integrals using the residues.
    \item Understand Cauchy’s integral formulas and there consequences including the fundamental theorem of algebra.
MATH 390 – Research Experience in Mathematics     1 credit hour
This course provides opportunities for students to reflect on their experience in undergraduate research activities and meet the Experiential Learning requirement for degrees in mathematics. Concurrent participation in the Undergraduate Research Fellows program at UNK or recent participation in a similar research program such as Summer Student Research Program at UNK or Research Experiences for Undergraduates at any institution is required for enrollment in this course.
Total Credits Allowed: 4.00
Prerequisite: Acceptance to the Undergraduate Research Fellows program or department permission.


MATH 399 – Internship     1-4 credit hours
On the job experience designed to complement the major. Internship experiences are available only in selected areas. Consult with the departmental advisor. MATH 399 is a credit/no credit course.
Total Credits Allowed: 4.00
MATH 400 – History of Mathematics     3 credit hours
An introduction to the history of mathematics from its primitive origins to modern-day mathematics.
Prerequisite: MATH 115

Learning Outcomes for MATH 400
On completion of MATH 400, students will 
    \item understand the progression of mathematics through history.
    \item understanding of the history of a variety of mathematical topics.
    \item gain an understanding of several mathematicians through history.
MATH 404 – Theory of Numbers     3 credit hours
Properties of integers, congruencies, primitive roots, arithmetic functions, quadratic residues, and the sum of squares.
Prerequisite: MATH 250 or permission of instructor.

Learning Outcomes for MATH 404
Upon successful completion of this course, students will be able to
    \item explain the concepts of divisibility, prime number, and congruence, 
    \item calculate the great common divisor using the Euclidean algorithm and prime factorization, 
    \item solve linear congruence and quadratic congruence, 
    \item understand Wilson’s Theorem and Fermat’s Little Theorem, 
    \item compute Euler's torsion function and other important multiplicative functions,
    \item use primitive roots and index arithmetic to solve higher-order congruence,  
    \item solve linear Diophantine equations and find primitive Pythagorean triples,
    \item understand how rational numbers are related to repeating decimals and continued fractions.

MATH 413 – Discrete Mathematics     3 credit hours
Topics include mathematical induction, recursion relations, counting principles, and discrete probability. Additional topics may include graph theory.
Prerequisite: MATH 250

Learning Outcomes for MATH 413
On completion of MATH 413, students will 
    \item gain an understanding of counting principles and how to apply them.
    \item gain an understanding of discrete structures and how to use and analyze them, including induction, recursion, and probabilistic methods.

MATH 420 – Numerical Analysis     3 credit hours
The solution of nonlinear equations, interpolation and approximation, numerical integration, matrices and system of linear equations, and numerical solution of differential equations.
Prerequisite: MATH 260 or permission of instructor.

Learning Outcomes MATH 420
On completion of MATH 420, students will
    \item understand IEEE arithmetic and know the rules for accurate computation.
    \item be able to determine the time complexity of basic algorithms.
    \item understand the concepts of linear and quadratic convergence and use these concepts to analyze the efficiency of an algorithm.
    \item develop a understanding of the algorithms for solving linear and nonlinear equations, interpolation, quadrature, and solution of differential equations.
    \item be able to use software tools, including graphical tools, to use these algorithms to solve problems numerically.
MATH 430 – Middle School Mathematics     3 credit hours
Topics will build on the foundations of MATH 230 and MATH 330 be focused toward the middle school math curriculum: algebraic structures including variables and functions, introductory number theory, probability, statistics, geometry, and problem solving.
Prerequisite: MATH 115 or MATH 202 or MATH 230 or MATH 260.
Learning Outcomes for MATH 430
Upon completion of this course, students will be able to: 

· Conceptualize the real number system, including rational and irrational numbers 
· Explain algebraic procedures (i.e. solving equations/inequalities, laws of exponents) 
· Simplify exponential and radical expressions 
· Apply transformation properties of congruent and similar figures 
· Identify and interpret graphs and functions 
· Compare and contrast different types of functions 
· Apply the Pythagorean Theorem to variety of situations 
· Use the coordinate plane to solve problems and display mathematics 
· Calculate, display and interpret statistical measures 
· Perform probability simulations and interpret results 
· Approach mathematical problems using a variety of methods. 
· Use various teaching models and techniques of curriculum delivery including effective questioning, cooperative learning, inquiry, and constructivist learning. 
· Explain mathematical concepts to students at their level of understanding. 
· Be reflective of own learning and realize how his/her own understanding influences student learning.

MATH 445 – Actuarial Science Seminar     1 credit hour
The purpose of this course is to develop knowledge of the fundamental probability tools for quantitatively assessing risk. The application of these tools to problems encountered in actuarial science is emphasized for the preparation of taking the exam P1. A thorough command of the supporting calculus is assumed, as well as exposure to many of the probability topics covered in STAT 441.
Prerequisite: STAT 441
MATH 460 – Advanced Calculus I     3 credit hours
Functions, sequences, limits, continuity, differentiation and integration.
Prerequisite: MATH 250 and MATH 260

Learning Outcomes for MATH 460
On completion of MATH 460, students will
    \item be able to prove basic propositions that involve the fundamentals of point set topology, including the concepts of open sets, closed sets, boundary points, limit points, and the set supremum and infimum. 
    \item demonstrate competence with basic properties of sequences including determining convergence and proving results involving the sum, difference, product, and quotient of sequences
    \item be able to use the definitions of continuity, the limit, and the derivative to prove basic propositions involving these concepts as well as be able to prove facts about specific functions.
    \item demonstrate the ability to the Mean Value Theorem to prove other theorems. 
    \item be able to define and compute with a variety of Riemann sums, including the lower, upper, and general sums.
    \item demonstrate a solid understanding of the fundamental theorem of calculus.

MATH 465 – Advanced Study in 9-12 Mathematics     2 credit hours
This course is an in-depth study of the 9-12 mathematics curriculum with a focus on mathematical practices, essential understandings, and connections with advanced mathematics. Preservice teachers will strengthen their conceptual understanding of number theory, algebra, calculus, probability, and statistics concepts in the 9-12 curriculum. They will also work on communicating mathematical ideas to secondary students. Throughout the course they will draw connections between math concepts learned in 9-12 grades and advanced mathematical topics from undergraduate studies.
Prerequisite: MATH 350 and MATH 430

Learning Outcomes, MATH 465
Upon completion of this course, students will be able to:
\itemConnect higher level content knowledge to essential content in secondary mathematics
\itemExplain the impact of higher-level mathematical content knowledge on their teaching of high school students
\itemArticulate and utilize mathematical practices essential for 9-12 mathematics.
\itemDescribe essential understandings for 9-12 students in number theory, algebra/functions. statistics, probability, and calculus
\itemDemonstrate the interconnectedness of mathematics among mathematical ideas
\itemUtilize technological tools to explore essential mathematical content in number/quantity, algebra. statistics/probability, and calculus

MATH 470 – Methods in Middle and High School Mathematics Teaching II     2 credit hours
In this second methods course, preservice teachers develop specialized research-based knowledge and instructional practices that facilitate mathematics learning for grades 6-12 students. The topics investigated in the course include mathematics research literature, differentiation, diversity and equity, assessment practices, and the development of effective mathematics lesson plans and curricular units. In addition, preservice teachers examine the importance of continuously improving teaching of mathematics through teacher reflection, instructional leadership, and professional development. MATH 471, a corequisite course, provides the opportunity to put learning into practice.
Prerequisite: MATH 270 and MATH 271 and TE 319 and TE 320 or TE 472 and TE 473. 
Corequisite: MATH 471.

MATH 470 Learning Outcomes
Upon completion of this course, students will be able to:
\item Develop effective unit and lessons that support district, state and national standards and are developmentally appropriate.
\item Incorporate various forms of communication and connections (including within the subject area, to other disciplines and to real life) into lessons.
\item Diagnose and assess student performance in a variety of ways, including formative, summative, open ended and performance assessments.
\item Address student diversity and various learning needs in lessons and units.
\item Use teaching methods and techniques of curriculum delivery including effective questioning, cooperative learning, inquiry, technology and problem solving
\item Incorporate various classroom organization and management techniques when teaching students.
\item Develop mathematical experiences for students, which will lead to positive dispositions toward math.
 \item Explain the organization and benefits of the NCTM Standards, the Common Core State Standards, Nebraska State Standards, and current trends in mathematics education.
\item Define NCTM and NATM, and explain membership benefits associated with each organization, and articulate the importance of professional affiliations.
\item Be reflective of own learning and realize how his/her own understanding influences student learning.

Learning Outcomes MATH 470 (alternate)

MATH 471 – Field Experience in Middle and High School Mathematics II     1 credit hour
This 50 clock-hour mathematics specific field-based experience is designed to provide students advanced practice in classroom teaching. Under the mentorship of a practicing 6-12 grade mathematics teacher and the supervision of a UNK mathematics educator, preservice teachers will actively engage in the teaching of mathematics to 6-12 grade students.
Prerequisite: MATH 270 and MATH 271 and TE 319 and TE 320 or TE 472 and TE 473. 
Corequisite: MATH 470.

Learning Outcomes, MATH 471

COURSE OBJECTIVES:Students will:
\itemUtilize research-based mathematics teaching practices in the classroom
\itemEngage 6-12 students in developmentally appropriate mathematics lessons
\itemIncorporate technology and tools into the 6-12 class in order to enhance mathematical understanding
\itemWork with a diverse range of students individually, in small groups, and in large class settings
\itemPlan, facilitate, and reflect upon mathematics lessons that promote reasoning and sense making
\itemCollect and analyze data to determine if 6-12 students have built new knowledge 
\itemCollaborate with colleagues, other school professionals, families, and stakeholders
\itemContinue to develop as a reflective practitioner
\itemMeet expectations of all Teacher Education Dispositions, including:
oDemonstrate effective oral communication skills
oDemonstrate effective written communication skills
oDemonstrate professionalism
oDemonstrate a positive and enthusiastic attitude
oDemonstrate preparedness in teaching and learning
oExhibits an appreciation of and value for cultural and academic diversity
oCollaborates effectively with stakeholders
oDemonstrates self-regulated learner behaviors/takes initiative
oExhibits the social and emotional intelligence to promote personal and educational goals/stability

MATH 490 – Special Topics in Mathematics     1-3 credit hours
Topics chosen from the areas of mathematics appropriate to the student's program and will involve both formal lectures and independent study.
Total Credits Allowed: 3.00
Learning Outcomes for MATH 490
The learning outcomes for MATH 490 vary by the course content.

MATH 495 – Independent Study in Mathematics     1-3 credit hours
An individual investigation by the student of topics not included in the normal mathematics offerings.
Department Consent Required
Total Credits Allowed: 5.00
Prerequisite: MATH 260
Learning Outcomes for MATH 495
The learning outcomes for MATH 495 vary by the course content.
MATH 496 – Mathematics Seminar     1 credit hour
Topics not included in the normal mathematics offerings are presented by the students.
Prerequisite: MATH 260 or permission of instructor.

Learning Outcomes for MATH 496
The learning outcomes for MATH 496 vary by the course content.

STAT 235 – Introduction to Statistics for Social Sciences 3 credit hours
An introduction to statistics for educational and sociological research. The course will include descriptive statistics, normal distribution and an introduction to correlation and hypothesis testing.
Prerequisite: Completion of MATH 101 or MATH 102 or MATH 115 or MATH 123 or Math ACT score of 20 or greater Students may not enroll in STAT 235 after earning credit for STAT 241.
STAT 241 – Elementary Statistics 3 credit hours
An introduction to statistics for sciences and business. The course will include graphing techniques, descriptive statistics, elementary probability models, estimation and hypothesis testing, and an introduction to correlation and regression.
Prerequisite: MATH 101 or MATH 102 or MATH 123 or MATH 115 or ACT Math score of 20 or greater
STAT 345 – Applied Statistics I 3 credit hours
Descriptive statistics; statistical inference using the binomial, normal, F and Chi Square distributions; and analysis of variance topics. Recommended for departmental majors as the beginning applied statistics course.
Prerequisite: MATH 115 or MATH 123
STAT 399 – Internship 1-4 credit hours
On the job experience designed to complement the major. Internships are available only in selected areas. Consult with departmental advisor. (Credit/No Credit)
Total Credits Allowed: 4.00
Prerequisite: MATH 115 or MATH 123
STAT 441 – Probability and Statistics 3 credit hours
The mathematical development of discrete and continuous probability distributions including multivariate distributions, moments and moment generating functions, the special discrete and continuous probability distributions, the normal distribution, sampling distributions, and hypothesis testing.
Prerequisite: MATH 260
STAT 442 – Mathematical Statistics 3 credit hours
A continuation of STAT 441. The further mathematical development of special probability densities, functions of random variables, sampling distributions, decision theory, point and interval estimators, hypotheses testing, and covariance.
Prerequisite: STAT 441
STAT 495 – Independent Study in Statistics 1-3 credit hours
An individual investigation by the student of topics not included in the normal statistics offerings.
Department Consent Required
Total Credits Allowed: 3.00

Learning Outcomes for STAT 495

The learning outcomes for STAT 495 vary by the course content.


